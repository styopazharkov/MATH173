
\documentclass{article}
\usepackage{amsmath, amsfonts, amssymb, amsthm} %packages for math-related
\usepackage[margin=1.2in]{geometry}
\usepackage[shortlabels]{enumitem}
\usepackage[utf8]{inputenc}
\usepackage{graphicx} % package for inserting images
\usepackage{commath} % package for things like \del, \cbr, and \sbr. These handle parentheses well.
\usepackage[mathscr]{euscript}%for \scr command 
\usepackage{../commands} %package with all of the commands for this class
\usepackage{url}
\setlength{\parindent}{0em} % so paragraphs aren't indented
\newcommand{\lcm}{\text{lcm}}
\newcommand{\Hom}{\text{Hom}}
\newcommand{\Ann}{\text{Ann}}
\newcommand{\Cc}{{C_c^\infty(\RR)}}
% ********************************************************** %
%               THINGS TO EDIT BELOW THIS LINE               %
% ********************************************************** %
\newcommand{\D}{\nabla}
\renewcommand{\d}{\partial}
\usepackage{wrapfig}
\title{\textsc{MATH 173 Problem Set 9}}
\author{Stepan (Styopa) Zharkov}
\date{June 1, 2022}
\begin{document}
\maketitle
\problem{1} 
\tri
\hop 
\solution

\newpage
\problem{2} 
\tri
\hop 
\solution
\begin{enumerate}[(a)]
    \item We need $\intzo |x^\alpha|^2 = \intzo x^{2\alpha}$ to converge. This converges for $\alpha > -1/2$ and diverges for $\alpha \le -1/2$, so $\phi_\alpha \in L^2((0,1))$ for $\alpha > -1/2$. \qed
    \item We need $\phi_\alpha \in L^2((0,1))$, so $\alpha > -1/2$. But, since $\phi_\alpha$ are smooth, we also need $\intzo |\phi_\alpha ' |^2$ to converge. We see $\phi_\alpha' = \alpha x^{\alpha-1}$. and $\intzo |\alpha x^{\alpha-1}|^2 = |\alpha|^2\intzo x^{2(\alpha-1)}$ converges for $\alpha > 1/2$ and diverges for $\alpha \le 1/2$. So, $\phi_\alpha \in H^1((0,1))$ for $\alpha > 1/2$. \qed
\end{enumerate}

\newpage
\problem{3} 
\tri
\hop 
\solution
\begin{enumerate}[(a)]
    \item We know the statement is true for $f \in C^1((a,b))$ by FTC.
    \item TODO
\end{enumerate}

\newpage
\problem{4} 
\tri
\hop 
\solution

\newpage
\problem{5} 
\tri
\hop 
\solution

\newpage
\problem{6} 
\tri
\hop 
\solution

\newpage
\problem{7} 
\tri
\hop 
\solution
\end{document}