
\documentclass{article}
\usepackage{amsmath, amsfonts, amssymb, amsthm} %packages for math-related
\usepackage[margin=1.2in]{geometry}
\usepackage[shortlabels]{enumitem}
\usepackage[utf8]{inputenc}
\usepackage{graphicx} % package for inserting images
\usepackage{commath} % package for things like \del, \cbr, and \sbr. These handle parentheses well.
\usepackage[mathscr]{euscript}%for \scr command 
\usepackage{../commands} %package with all of the commands for this class
\usepackage{url}
\setlength{\parindent}{0em} % so paragraphs aren't indented
\newcommand{\lcm}{\text{lcm}}
\newcommand{\Hom}{\text{Hom}}
\newcommand{\Ann}{\text{Ann}}
\newcommand{\Cc}{{C_c^\infty(\RR)}}
% ********************************************************** %
%               THINGS TO EDIT BELOW THIS LINE               %
% ********************************************************** %
\newcommand{\D}{\nabla}
\renewcommand{\d}{\partial}
\usepackage{wrapfig}
\title{\textsc{MATH 173 Problem Set 9}}
\author{Stepan (Styopa) Zharkov}
\date{June 1, 2022}
\begin{document}
\maketitle
\problem{1} 
\tri
\hop 
\solution

\newpage
\problem{2} 
\tri
\hop 
\solution
\begin{enumerate}[(a)]
    \item We need $\intzo |x^\alpha|^2 = \intzo x^{2\alpha}$ to converge. This converges for $\alpha > -1/2$ and diverges for $\alpha \le -1/2$, so $\phi_\alpha \in L^2((0,1))$ for $\alpha > -1/2$. \qed
    \item We need $\phi_\alpha \in L^2((0,1))$, so $\alpha > -1/2$. But, since $\phi_\alpha$ are smooth, we also need $\intzo |\phi_\alpha ' |^2$ to converge. We see $\phi_\alpha' = \alpha x^{\alpha-1}$. and $\intzo |\alpha x^{\alpha-1}|^2 = |\alpha|^2\intzo x^{2(\alpha-1)}$ converges for $\alpha > 1/2$ and diverges for $\alpha \le 1/2$. So, $\phi_\alpha \in H^1((0,1))$ for $\alpha > 1/2$. \qed
\end{enumerate}

\newpage
\problem{3} 
\tri
\hop 
\solution
\begin{enumerate}[(a)]
    \item We know the statement is true for $f \in C^1((a,b))$ by FTC. Now, let $f_n \to f$ where $f_n \in C^1((a,b))$. By the continuity of the trace operator, 
    \[f(x) - f(y) = \limty (f_n(x)-f_n(y)) = \limty \int_x^yf_n'(t)dt\]
    Now, since we are on a bounded interval, we can move the limit inside the derivative after applying dominated convergence to see that 
    \[\limty \int_x^yf'_n(t)dt - \int_x^yf'_n(t)dt = \limty \int_x^y(f_n(t) - f(t))'dt = 0.\]
    So, 
    \[f(x) - f(y) =  \int_x^yf'(t)dt\]
    as we wanted. \qed
    \item TODO
\end{enumerate}

\newpage
\problem{4} 
\tri
\hop 
\solution

\newpage
\problem{5} 
\tri
\hop 
\solution Consider the dogbowl functions 
\[f_n := 1-\min(n\cdot d(x, \d B), 1).\]

\image{4cm}{dogbowl}

We see that $Tf_n = 1$ for all $n$, so $Tf_n \to 1 \ne 0$. However, 
\[||f_n||_{L_2}^2 = \int_B |f_n(x)|^2dx = \int_{d(x,\d B)< 1/n} |f_n(x)|^2dx \le  \int_{d(x,\d B)< 1/n} 1 dx  = O(1/n) \to 0.\]
So, $f_n \to 0$ in $L^2$. \qed
 
\newpage
\problem{6} 
\tri
\hop 
\solution Let $u = \limty u_n$ where $u_n$ are compactly supported continuous functions. Note that we are given that $u = \limty -u_n(x^*)$. This means that 
\[u = \frac{ \limty u_n(x) + \limty -u_n(x^*)}{2}= \limty \frac{u_n(x)-u_n(x^*)}{2}.\]
Note that $\frac{u_n(x)-u_n(x^*)}{2}= 0$ when $x_n =0$, so 
\[T_{B_+}\del{\frac{u_n(x)-u_n(x^*)}{2}} =\frac{u_n(x)-u_n(x^*)}{2}|_{\d B_+} = 0.\]
We have shown in class that this is sufficient to say $T_{B_+}(u|_{B_+}) = 0$, so $u|_{B_+}\in H_0^1(B_+)$. \qed

\newpage
\problem{7} 
\tri
\hop 
\solution
\begin{enumerate}[(a)]
    \item Let $V_k = \{x : |x| < 1/k\}$ and let $W_k =\{x: 1- 1/k < |x| < 1\}$. Then, consider lemonsqueezer functions $f_k \in $ such that $f \in C_0^1(U)$ and $f_k|_{B(V_k \cap W_k)} = 1$ and $f_k|_{V_{2k} \cup W_{2k}} = 0$. 
    
    TODO: image

    Note that $u_k := uf_k \in C_0^1(U)$. We claim $u_k \to u$ in $H^1(B)$. Note that 
    \[||u-u_k||_{H^1}^2 = \int_B|u - u_k|^2+ \int_B|\D u -\D u_k|^2.\] 
    Now, since $u$ is bounded
    \begin{align*}
        \int_B |u - u_k|^2 &= \int_B|u|^2|1-f_k|^2 \\
        &= \int_BO(1)|1-f_k|^2 \\
        &=O(1) \int_B|1-f_k|^2 \\
        &=O(1) \int_{V_k \cup W_k}|1-f_k|^2 \\
        &=O(1) \int_{V_k \cup W_k}O(1) \\
        &=O(1/k^n)=o(1)
    \end{align*}
    where the asymptotic notation is with respect to $k$.
    Also, 
    \begin{align*}
        \int_B|\D u -\D u_k|^2 = \int_B|\D u -\D u f_k - u \D f_k |^2 \le 2\int_B|\D u|^2 |1-f_k|^2 + 2\int_B|u \D f_k|^2.
    \end{align*}
    Note that since $|\D u|$ is bounded, applying our above logic to the first part gives us 
    \[2\int_B|\D u|^2 |1-f_k|^2 = O(1) \int_B|1-f_k|^2 = o(1).\]
    So, we only need to deal with the second part. We see that since $u$ is bounded and $\D f_k$ is mostly 0,
    \begin{align*}
        2\int_B|u \D f_k|^2 &=  2\int_B|u|^2|\D f_k|^2\\
        &= O(1)\int_B |\D f_k|^2 \\
        &= O(1)\int_{V_k \cup W_k}|\D f_k|^2 \\
        &= O(1)\int_{V_k \cup W_k}|O(k)|^2 \\
        &= O(1) O(1/k^n) O(k^2) \\
        &= o(1)
    \end{align*}
    for $n > 2$. Thus, combining everything, we see that 
    \[||u-u_k||_{H^1}^2 = o(1),\]
    which is what we needed to show that $H_0^1(B) = H_0^1(U)$. \qed
    \item Consider $u(x) = 1- x^2 \in C^1((-1,1))$. Note that $T_{(-1,1)} u = 0$, so $u \in H_0^1((-1,1))$. However, $T_{(-1,0) \cup (0,1)}u \ne0 $ so $u \not \in H_0^1((-1,0) \cup (0,1))$. Thus, 
    \[ H_0^1((-1,1)) \ne H_0^1((-1,0) \cup (0,1))\]
    \qed
    \item TODO
\end{enumerate}
\end{document}