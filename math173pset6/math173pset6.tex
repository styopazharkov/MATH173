
\documentclass{article}
\usepackage{amsmath, amsfonts, amssymb, amsthm} %packages for math-related
\usepackage[margin=1.2in]{geometry}
\usepackage[shortlabels]{enumitem}
\usepackage[utf8]{inputenc}
\usepackage{graphicx} % package for inserting images
\usepackage{commath} % package for things like \del, \cbr, and \sbr. These handle parentheses well.
\usepackage[mathscr]{euscript}%for \scr command 
\usepackage{../commands} %package with all of the commands for this class
\usepackage{url}
\setlength{\parindent}{0em} % so paragraphs aren't indented
\newcommand{\lcm}{\text{lcm}}
\newcommand{\Hom}{\text{Hom}}
\newcommand{\Ann}{\text{Ann}}
\newcommand{\Cc}{{C_c^\infty(\RR)}}
% ********************************************************** %
%               THINGS TO EDIT BELOW THIS LINE               %
% ********************************************************** %
\newcommand{\D}{\nabla}
\renewcommand{\d}{\partial}
\usepackage{wrapfig}
\title{\textsc{MATH 173 Problem Set 6}}
\author{Stepan (Styopa) Zharkov}
\date{May 11, 2022}
\begin{document}
\maketitle
\problem{1} TODO
 \tri
\hop 
\solution
\begin{enumerate}[(a)]
    \item This problem is straightforward. 
    \begin{align*}
        \ol{\F(\phi)(y)} &= \ol {\int_\Rn e^{-ix \cdot y}\phi(x)dx} \\
        &= \int_\Rn \ol {e^{-ix \cdot y}\phi(x)}dx \\
        &= \int_\Rn e^{ix \cdot y}\ol{\phi(x)}dx \\
        &= (2\pi)^n(2\pi)^{-n}\int_\Rn e^{ix \cdot y}\ol{\phi(x)}dx \\
        &= (2\pi)^n\F\inv\ol{\phi}(y) 
    \end{align*}
    This is what we wanted to show. \qed 
    \item We know the Fourier inversion formula holds for Schwartz functions. By part (a) and the equation about interchanging fourier transform under the integral that we saw in class, we know 
    \begin{align*}
        (2\pi)^{-n}\int_\Rn \hat{\phi}\bar{\hat{\psi}} &= \int_\Rn \hat{\phi}\check{\bar{\psi}}  \\
        (2\pi)^{-n}\int_\Rn \hat{\phi}\bar{\hat{\psi}} &= \int_\Rn \check{\hat{\phi}}\hat{\check{\bar{\psi}}}  \\
        (2\pi)^{-n}\int_\Rn \hat{\phi}\bar{\hat{\psi}} &= \int_\Rn \phi\bar{\psi}.
    \end{align*}
    Setting $\psi = \phi$, we see that 
    \begin{align*}
        \int_\Rn \abs{\hat{\phi}}^2 = (2\pi)^2\int_\Rn \abs{{\phi}}^2,
    \end{align*}
    as we wanted. \qed
\end{enumerate}



\newpage
\problem{2} TODO
 \tri
\hop 
\solution
\newcommand{\tu}{\tilde{u}}
\begin{enumerate}[(a)]
    \item Let $\tilde{u}(t,x)$ be defined as in the problem. On $(0, +\infty) \times (0, +\infty)$, $\tilde{u}$ is the same as $u$, so it satisfies the equation $\tu_t = \tu_{xx}$. On $(0, +\infty) \times (-\infty, 0)$, we see that 
    \[\tu_t(t,x) = -u_t(t, -x) = -u_{xx}(t, -x) = \tu_{xx}(t,x).\]
    So, $\tu_t = \tu_{xx}$ on all of  $(0, +\infty) \times (0, +\infty)  \cup (0, +\infty) \times (-\infty, 0)$ and is $C^2$ there. Now, consider the points along $x=0$. We can define $\tu(t,0)= 0$. We see that 
    \[\lim_{x \to 0} u(t,x) = u(t,0) = 0,\]
    and thus $\lim_{x \to 0} \tu(t,x) = 0$ from both sides, and is equal to $\tu(t,0)$. So, $\tu$ is continuous in $[0,+\infty)\times \RR$. 
    \hop 
    Now, let's consider differentiability. Let $t > 0$. We see that 
    \begin{align*}
        \lim_{h\to^+ 0} \frac{\tu(t,h)- \tu(t,0)}{h} &=  \lim_{h\to^+ 0} \frac{\tu(t,h)}{h} \\
        &= \lim_{h\to^+ 0} \frac{u(t,h)}{h}\\
        &= \lim_{h\to^- 0} \frac{u(t,-h)}{-h}\\
        &= \lim_{h\to^- 0} \frac{\tu(t,h)}{h}.\\
    \end{align*}
    Thus, the derivative from both sides matches up. We see that $\tu_t(t,0) = 0$. Since $u$ is continuously differentiable on the border, both components of the derivative are continuous, so $u$ is differentiable on $(0, +\infty) \times \RR$. 
    \hop 
    We can now assume that $\tu = K_t * \tilde{g}$. Writing this out, we have 
    \begin{align*}
        \tu(t,x) &= (K_t * \tilde{g})(x)\\
        &= \int(4\pi t)^{1/2} e^{-\frac{|x-y|^2}{4t}} \tilde{g}(y)dy\\
        &= \sqrt{4\pi t}\int_{0}^\infty  e^{-\frac{|x-y|^2}{4t}} \tilde{g}(y)dy+\sqrt{4\pi t}\int_{-\infty}^0  e^{-\frac{|x-y|^2}{4t}} \tilde{g}(y)dy\\
        &= \sqrt{4\pi t}\int_{0}^\infty  e^{-\frac{|x-y|^2}{4t}} \tilde{g}(y)dy+\sqrt{4\pi t}\int_{-\infty}^0  e^{-\frac{|x-y|^2}{4t}} \tilde{g}(y)dy\\
    \end{align*} 
    \item TODO
\end{enumerate}


\newpage
\problem{3} TODO
 \tri
\hop 
\solution
TODO 


\newpage
\problem{4} TODO
 \tri
\hop 
\solution
TODO 


\newpage
\problem{5} TODO
 \tri
\hop 
\solution
TODO 


\newpage
\problem{6} TODO
 \tri
\hop 
\solution
TODO 


\newpage
\problem{7} TODO
 \tri
\hop 
\solution
TODO 

\end{document}