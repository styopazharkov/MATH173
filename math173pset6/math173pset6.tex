
\documentclass{article}
\usepackage{amsmath, amsfonts, amssymb, amsthm} %packages for math-related
\usepackage[margin=1.2in]{geometry}
\usepackage[shortlabels]{enumitem}
\usepackage[utf8]{inputenc}
\usepackage{graphicx} % package for inserting images
\usepackage{commath} % package for things like \del, \cbr, and \sbr. These handle parentheses well.
\usepackage[mathscr]{euscript}%for \scr command 
\usepackage{../commands} %package with all of the commands for this class
\usepackage{url}
\setlength{\parindent}{0em} % so paragraphs aren't indented
\newcommand{\lcm}{\text{lcm}}
\newcommand{\Hom}{\text{Hom}}
\newcommand{\Ann}{\text{Ann}}
\newcommand{\Cc}{{C_c^\infty(\RR)}}
% ********************************************************** %
%               THINGS TO EDIT BELOW THIS LINE               %
% ********************************************************** %
\newcommand{\D}{\nabla}
\renewcommand{\d}{\partial}
\usepackage{wrapfig}
\title{\textsc{MATH 173 Problem Set 6}}
\author{Stepan (Styopa) Zharkov}
\date{May 11, 2022}
\begin{document}
\maketitle
\problem{1} TODO
 \tri
\hop 
\solution
\begin{enumerate}[(a)]
    \item This problem is straightforward. 
    \begin{align*}
        \ol{\F(\phi)(y)} &= \ol {\int_\Rn e^{-ix \cdot y}\phi(x)dx} \\
        &= \int_\Rn \ol {e^{-ix \cdot y}\phi(x)}dx \\
        &= \int_\Rn e^{ix \cdot y}\ol{\phi(x)}dx \\
        &= (2\pi)^n(2\pi)^{-n}\int_\Rn e^{ix \cdot y}\ol{\phi(x)}dx \\
        &= (2\pi)^n\F\inv\ol{\phi}(y) 
    \end{align*}
    This is what we wanted to show. \qed 
    \item We know the Fourier inversion formula holds for Schwartz functions. By part (a) and the equation about interchanging fourier transform under the integral that we saw in class, we know 
    \begin{align*}
        (2\pi)^{-n}\int_\Rn \hat{\phi}\bar{\hat{\psi}} &= \int_\Rn \hat{\phi}\check{\bar{\psi}}  \\
        (2\pi)^{-n}\int_\Rn \hat{\phi}\bar{\hat{\psi}} &= \int_\Rn \check{\hat{\phi}}\hat{\check{\bar{\psi}}}  \\
        (2\pi)^{-n}\int_\Rn \hat{\phi}\bar{\hat{\psi}} &= \int_\Rn \phi\bar{\psi}.
    \end{align*}
    Setting $\psi = \phi$, we see that 
    \begin{align*}
        \int_\Rn \abs{\hat{\phi}}^2 = (2\pi)^2\int_\Rn \abs{{\phi}}^2,
    \end{align*}
    as we wanted. \qed
\end{enumerate}



\newpage
\problem{2} TODO
 \tri
\hop 
\solution
\newcommand{\tu}{\tilde{u}}
\begin{enumerate}[(a)]
    \item Let $\tilde{u}(t,x)$ be defined as in the problem. On $(0, +\infty) \times (0, +\infty)$, $\tilde{u}$ is the same as $u$, so it satisfies the equation $\tu_t = \tu_{xx}$. On $(0, +\infty) \times (-\infty, 0)$, we see that 
    \[\tu_t(t,x) = -u_t(t, -x) = -u_{xx}(t, -x) = \tu_{xx}(t,x).\]
    So, $\tu_t = \tu_{xx}$ on all of  $(0, +\infty) \times (0, +\infty)  \cup (0, +\infty) \times (-\infty, 0)$ and is $C^2$ there. Now, consider the points along $x=0$. We can define $\tu(t,0)= 0$. We see that 
    \[\lim_{x \to 0} u(t,x) = u(t,0) = 0,\]
    and thus $\lim_{x \to 0} \tu(t,x) = 0$ from both sides, and is equal to $\tu(t,0)$. So, $\tu$ is continuous in $[0,+\infty)\times \RR$. 
    \hop 
    Now, let's consider differentiability. Let $t > 0$. We see that 
    \begin{align*}
        \lim_{h\to^+ 0} \frac{\tu(t,h)- \tu(t,0)}{h} &=  \lim_{h\to^+ 0} \frac{\tu(t,h)}{h} \\
        &= \lim_{h\to^+ 0} \frac{u(t,h)}{h}\\
        &= \lim_{h\to^- 0} \frac{u(t,-h)}{-h}\\
        &= \lim_{h\to^- 0} \frac{\tu(t,h)}{h}.\\
    \end{align*}
    Thus, the derivative from both sides matches up. We see that $\tu_t(t,0) = 0$. Since $u$ is continuously differentiable on the border, both components of the derivative are continuous, so $u$ is differentiable on $(0, +\infty) \times \RR$. 
    \hop 
    We can now assume that $\tu = K_t * \tilde{g}$. Writing this out, we have 
    \begin{align*}
        \tu(t,x) &= (K_t * \tilde{g})(x)\\
        &= \int(4\pi t)^{-1/2} e^{-\frac{|x-y|^2}{4t}} \tilde{g}(y)dy\\
        &=  \frac{1}{\sqrt{4\pi t}}\int_{0}^\infty  e^{-\frac{|x-y|^2}{4t}} \tilde{g}(y)dy+ \frac{1}{\sqrt{4\pi t}}\int_{-\infty}^0  e^{-\frac{|x-y|^2}{4t}} \tilde{g}(y)dy\\
        &=  \frac{1}{\sqrt{4\pi t}}\int_{0}^\infty  e^{-\frac{(x-y)^2}{4t}} g(y)dy- \frac{1}{\sqrt{4\pi t}}\int_{0}^\infty  e^{-\frac{(x+y)^2}{4t}} g(y)dy\\
        &=  \frac{1}{\sqrt{4\pi t}}\int_{0}^\infty  g(y)\del{ e^{-\frac{(x-y)^2}{4t}} - e^{-\frac{(x+y)^2}{4t}}}dy.\\
    \end{align*} 
    Restricting to to half, we see that 
    \[u(t,x) = \frac{1}{\sqrt{4\pi t}}\int_{0}^\infty  g(y)\del{ e^{-\frac{(x-y)^2}{4t}} - e^{-\frac{(x+y)^2}{4t}}}dy.\]
    We can verify that the boundary conditions match. \qed
    \item This is very similar to part (a), but this time let us define 
    \[\tu(t,x) = \begin{cases}
        u(t,x) &\text{ if } x \ge 0 \\
        u(t,-x) &\text{ if } x \ge 0 \\
    \end{cases}\]
    to be the even extension. 
    \hop 
    We know $\tu$ is $C^2$ and satisfies the equation on  $(0, +\infty) \times (0, +\infty)  \cup (0, +\infty) \times (-\infty, 0)$ for the same reason as in part (a). 
    \hop 
    We see that 
    \[\lim_{x \to 0} u(t,x) = u(t,0),\]
    and thus $\lim_{x \to 0} \tu(t,x) = u(t,0)$ from both sides, and is equal to $\tu(t,0)$. So, $\tu$ is continuous in $[0,+\infty)\times \RR$. 
    \hop 
    Now, we notice that 
    \begin{align*}
        \lim_{h\to^+ 0} \frac{\tu(t,h)- \tu(t,0)}{h}
        &= \lim_{h\to^+ 0} \frac{u(t,h) - u(t,0)}{h}\\
        &= 0 \\
        &= \lim_{h\to^+ 0} -\frac{u(t,h) - u(t,0)}{h}\\
        &= \lim_{h\to^- 0} -\frac{u(t,-h) - u(t,0)}{h}\\
        &= \lim_{h\to^- 0} \frac{\tu(t,h)-\tu(t,0)}{h}.\\
    \end{align*}
    Thus, the derivative from both sides matches up. We see that $\tu_t(t,0) = 0$. Since $u$ is continuously differentiable on the border, both components of the derivative are continuous, so $u$ is differentiable on $(0, +\infty) \times \RR$. 
    \hop 
    Making a similar assumption, we see that this time 
    \begin{align*}
        \tu(t,x) &= (K_t * \tilde{g})(x)\\
        &= \int(4\pi t)^{-1/2} e^{-\frac{|x-y|^2}{4t}} \tilde{g}(y)dy\\
        &=  \frac{1}{\sqrt{4\pi t}}\int_{0}^\infty  e^{-\frac{|x-y|^2}{4t}} \tilde{g}(y)dy+ \frac{1}{\sqrt{4\pi t}}\int_{-\infty}^0  e^{-\frac{|x-y|^2}{4t}} \tilde{g}(y)dy\\
        &=  \frac{1}{\sqrt{4\pi t}}\int_{0}^\infty  e^{-\frac{(x-y)^2}{4t}} g(y)dy+ \frac{1}{\sqrt{4\pi t}}\int_{0}^\infty  e^{-\frac{(x+y)^2}{4t}} g(y)dy\\
        &=  \frac{1}{\sqrt{4\pi t}}\int_{0}^\infty  g(y)\del{ e^{-\frac{(x-y)^2}{4t}} + e^{-\frac{(x+y)^2}{4t}}}dy.\\
    \end{align*} 
    Restricting to to half, we see that 
    \[u(t,x) = \frac{1}{\sqrt{4\pi t}}\int_{0}^\infty  g(y)\del{ e^{-\frac{(x-y)^2}{4t}} - e^{-\frac{(x+y)^2}{4t}}}dy.\]
    We can verify that the boundary condition matches. \qed
\end{enumerate}


\newpage
\problem{3} TODO
 \tri
\hop 
\solution
This problem is full of tricks and surprises. First, consider $v(t,x) = u(t,x) - a(t)$. This means that 
\[v_t = u_t - a'(t) = u_{xx} - a'(t)= v_xx - a'(t).\]
So, 
\[v_t - v_{xx} = -a'(t)\]
with $v(t,0)= 0$ and $v(0,x) = 0$. Also, $a(0)= 0$. So, we have an inhomogeneous heat equation.
\hop 
Let $S_t(\phi)$ be the operator in problem 2. More precisely, let 
\[S_t(\phi)(x) = \frac{1}{\sqrt{4\pi t}} \int_0^\infty \phi(y)\del{e^{-\frac{(x-y)^2}{4t}} - e^{-\frac{(x+y)^2}{4t}}}dy.\] 
Then, by Duhamel's principle, 
\[v(t,x) = \int_0^tS_{t-s}(f(s, \cdot))ds\]
where $f(t,x) = -a'(t)$. Expanding, and changing variables, we have 
\begin{align*}
    v(t,x) &= \int_0^t \frac{1}{\sqrt{4\pi (t-s)}} \int_0^\infty -a'(s)\del{e^{-\frac{(x-y)^2}{4(t-s)}} - e^{-\frac{(x+y)^2}{4(t-s)}}}dy ds \\
    &= \int_0^t \frac{-a'(s)}{\sqrt{4\pi (t-s)}} \sbr{\int_0^x e^{-\frac{(y)^2}{4(t-s)}}dy+\int_0^\infty e^{-\frac{(y)^2}{4(t-s)}}dy-\int_x^\infty e^{-\frac{(y)^2}{4(t-s)}}dy} ds\\
    &= \int_0^t \frac{-a'(s)}{\sqrt{4\pi (t-s)}} \sbr{\int_0^x e^{-\frac{(y)^2}{4(t-s)}}dy+\int_0^x e^{-\frac{(y)^2}{4(t-s)}}dy} ds\\
    &= \int_0^t \frac{-2a'(s)}{\sqrt{4\pi (t-s)}}\int_0^x e^{-\frac{(y)^2}{4(t-s)}}dy ds
\end{align*}
Now, using the hint, we can change variables and apply integration by parts to see that 
\begin{align*}
    v(t,x) &=  \int_0^t \frac{-2a'(s)}{\sqrt{4\pi (t-s)}}\int_0^{x(4(t-s))^{-1/2}} e^{-z^2}dz ds\\
    &=  \int_0^t \frac{-2a(s)}{\sqrt{\pi}} e^{\frac{-x^2}{4(t-s)}} \del{-(t-s)^{-3/2} \cdot \frac{1}{4}} ds\\
    &=  \frac{x^2}{\sqrt{4\pi}}\int_0^t (t-s)^{-3/2} e^{\frac{-x^2}{4(t-s)}}a(s) ds\\
    &=  \frac{x^2}{\sqrt{4\pi}}\int_0^t (s)^{-3/2} e^{\frac{-x^2}{4(s)}}a(t-s) ds.
\end{align*}
Thus, we can conclude that 
\[u(t,x)= a(t) + \frac{x^2}{\sqrt{4\pi}}\int_0^t (s)^{-3/2} e^{\frac{-x^2}{4(s)}}a(t-s) ds,\]
which is not exactly what we wanted, but I suspect there is a typo in the problem. \qed
\newpage
\problem{4} TODO
 \tri
\hop 
\solution
\begin{enumerate}[(a)]
    \item By d'Alembert's formula, we know 
    \begin{align*}
        u(x,t) &= \frac{1}{2}(\phi(x+ct)+\phi(x-ct)) + \frac{1}{2c}\int_{x-ct}^{x+ct} \psi(\sigma) d \sigma\\
        &= \frac{1}{2}((x+ct)^2+(x-ct)^2) + \frac{1}{2c}\int_{x-ct}^{x+ct} 1 d \sigma\\
        &= x^2 + (ct)^2 + t^2.
    \end{align*}
    \qed
    \item For this problem, we can repeatedly use the fundamental theorem of calculus. We assume that $u_x$ vanishes at infinity. We see 
    \begin{align*}
        \int u(t,x) dx &= \int \del{u(0,x) + \int_0^t u_t(\tau, x)d\tau} dx \\
        &= \int \del{u(0,x) + \int_0^t \del{u_t(0, x) +\int_0^\tau u_{tt}(s,x)ds}d\tau} dx\\
        &= \int \del{u(0,x) + \int_0^t \del{u_t(0, x) +\int_0^\tau c^2u_{xx}(s,x)ds}d\tau} dx\\
        &= \int u(0,x) dx  +\int \int_0^t u_t(0, x) d\tau dx+\int \int_0^t \int_0^\tau c^2u_{xx}(s,x)dsd\tau dx\\
        &= \int \phi(x) dx  +\int \int_0^t \psi(x) d\tau dx+\int \int_0^t \int_0^\tau c^2u_{xx}(s,x)dsd\tau dx\\
        &= \int \phi(x) dx  +t\int \psi(x) d\tau dx+ \int_0^t \int_0^\tau \int c^2u_{xx}(s,x) dx dsd\tau \\
        &= \int \phi(x) dx  +t\int \psi(x) d\tau dx+ \int_0^t \int_0^\tau 0\  dsd\tau \\
        &= \int \phi(x) dx  +t\int \psi(x) d\tau dx,
    \end{align*}
    exactly as we wanted. \qed
\end{enumerate}


\newpage
\problem{5} TODO
 \tri
\hop 
\solution
TODO 


\newpage
\problem{6} TODO
 \tri
\hop 
\solution
TODO 


\newpage
\problem{7} TODO
 \tri
\hop 
\solution
TODO 

\end{document}