
\documentclass{article}
\usepackage{amsmath, amsfonts, amssymb, amsthm} %packages for math-related
\usepackage[margin=1.2in]{geometry}
\usepackage[shortlabels]{enumitem}
\usepackage[utf8]{inputenc}
\usepackage{graphicx} % package for inserting images
\usepackage{commath} % package for things like \del, \cbr, and \sbr. These handle parentheses well.
\usepackage[mathscr]{euscript}%for \scr command 
\usepackage{../commands} %package with all of the commands for this class
\usepackage{url}
\setlength{\parindent}{0em} % so paragraphs aren't indented
\newcommand{\lcm}{\text{lcm}}
\newcommand{\Hom}{\text{Hom}}
\newcommand{\Ann}{\text{Ann}}
\newcommand{\Cc}{{C_c^\infty(\RR)}}
% ********************************************************** %
%               THINGS TO EDIT BELOW THIS LINE               %
% ********************************************************** %
\newcommand{\D}{\nabla}
\renewcommand{\d}{\partial}
\usepackage{wrapfig}
\title{\textsc{MATH 173 Problem Set 7}}
\author{Stepan (Styopa) Zharkov}
\date{May 18, 2022}
\begin{document}
\maketitle
\problem{1} 
 \tri
\hop 
\solution
This problem is straightforward. Since $\ol{\Omega}$ is closed, and $c > 0$, there exists constants $C_1, C_2 >0$ such that $C_1 < c(x) < C_2$. Assume $C_1 < 1$ and $C_2 > 1$. If not, we can always choose smaller $C_1$ and larger $C_2$. So, for any $u \in C^1(\ol{\Omega})$, 
\begin{align*}
    ||u||^2_{H^1_c(\Omega)} &= \int_\Omega |u(x)|^2 dx + \int_\Omega c(x) |\D u (x)|^2 dx \\
    &\ge C_1^2\int_\Omega |u(x)|^2 dx + \int_\Omega C_1^2|\D u (x)|^2 dx\\
    &= C_1^2||u||^2_{H^1(\Omega)}.
\end{align*}
Similarly, 
\begin{align*}
    ||u||^2_{H^1_c(\Omega)} &= \int_\Omega |u(x)|^2 dx + \int_\Omega c(x) |\D u (x)|^2 dx \\
    &\le C_2^2\int_\Omega |u(x)|^2 dx + \int_\Omega C_2^2|\D u (x)|^2 dx\\
    &= C_2^2||u||^2_{H^1(\Omega)}.
\end{align*}
So, the statement follows for all  $u \in C^1(\ol{\Omega})$. By continuity and density, it follows that the statement holds for $u \in H^1(\Omega)$. \qed
\newpage
\problem{2} 
 \tri
\hop 
\solution
\begin{enumerate}[(a)]
    \item Suppose $v = u+w = u' + w'$ where $u,u' \in M$, $w, w' \in M^\perp$. Then, we see 
    \[u' - u = u'+w - v = w - w'.\]
    But $u'-u \in M$ and $w - w' \in M^\perp$ and $M \cap M^\perp = \{0\}$. So, $u' - u = w-w' = 0$. Thus, the decomposition is unique. 
    \hop 
    We see that $u = u+0$. By uniqueness, $P(v) = u = P(u) = P(P(v))$, so $P = P^2$. \qed
    \item Let $v = u+w$ and $v' = u'+w'$ with $u,u' \in M$, $w, w' \in M^\perp$. We see 
    \begin{align*}
        \ang{Pv, v'}  &= \ang{u, l'+w'} = \ang{u, u'} + \ang{u,w'} = \ang{u, u'} = \ang{u,u'}+\ang{w, u'} = \ang{u+w, u'} = \ang{v, Pv'},
    \end{align*}
    So $P = P^*$ by definition. \qed 
    \item Since $T$ is linear, $T(H)$ is a subspace. For any sequence $v_j$ in $T(H)$ where $v_j \to v$ in $H$, we know $T(v_j) = v_j$ because $T^2 = T$. So, by continuity of $T$, we have $T(v) = v$, so $v \in T(H)$ and we can conclude $T(H)$ is closed. 
    \hop 
    Let $u = T(v)$ and let $w = v - u$. Note $u \in T(H)$. For any $y = T(x) \in T(H)$, we have 
    \[\ang{y, w} = \ang{T(x), v- u} = \ang{x, T(v-u)} = \ang{x, T(v)- T(u)} = \ang{x, u- u} = 0\]
    because $T = T^2 = T^*$. So, $w\in T(H)^\perp$ and thus $T$ is the projection. \qed
\end{enumerate}


\newpage
\problem{3} 
 \tri
\hop 
\solution
In this problem, we will use $H^1$ to denote $H^1((0,1))$ to reduce clutter.
\begin{enumerate}[(a)]
    \item Consider the operator $T: H^1 \to H^1$ where $Tf = f - \intzo f$. It's easy to see that $T$ is linear since integration is linear. We see that for any $f$, %TODO: be careful about the space
    \begin{align*}
        ||Tf||_{H^1} &= \intzo \abs{f - \intzo f}^2 + \intzo \abs{\D f}^2 \\
        &\le  \intzo \abs{\abs{f} - \abs{\intzo f}}^2 + \intzo \abs{\D f} ^2\\
        &\le  \intzo \del{2\abs{f}^2 +2 \abs{\intzo f}^2} + \intzo \abs{\D f}^2 \\
        &=  4\intzo \abs{f}^2 + \intzo \abs{\D f}^2 \\
        &\le  4\intzo \abs{f}^2 + 4\intzo \abs{\D f}^2 \\
        &= 4||f||_{H^1}.
    \end{align*}
    So, $T$ is bounded. It's easy to see that $T(f) \in M$ by definition, and if $f \in M$, then $T(f) = f$. So, $T = T^2$ and $T(H^1) = M$. Also, 
    \begin{align*}
        \ang{Tf, g} &= \ang{f-\intzo f, g } \\ 
        &= \intzo \del{f \ol{g} + \D f \ol{\D g}} - \intzo \del{\intzo f}\ol{g} \\
        &= \intzo \del{f \ol{g} + \D f \ol{\D g}} - \del{\intzo\ol{g}} \del{\intzo f} \\
        &= \intzo \del{f \ol{g} + \D f \ol{\D g}} - \intzo \del{\intzo \ol{g}}f \\
        &=\ang{f, g - \intzo g} \\
        &= \ang{f, Tg}.
    \end{align*}
    So, $T = T^*$. Now, by problem 2c, $M = T(H)$ is closed. 
    \hop 
    Since $T$ is the projection onto $M$, we know $M^\perp = \ker T$. That is $f$ such that $f - \intzo f = 0$. Thus, $f$ must be constant. We see that any constant function is in the kernel, so $M^\perp$ is the constant functions. \qed
    \item TODO
\end{enumerate}

\newpage
\problem{4} 
 \tri
\hop 
\solution
Let 
\[J^*g = \int_0^x\int_0^t g(s)dsdt - x\intzo\int_0^t g(s)ds dt. \]
It's easy to see that $J^*$ is linear. Now, we show it is bounded.
TODO: finish this
TODO: check boundedness
TODO: check adjointness

\newpage
\problem{5} 
 \tri
\hop 
\solution
\begin{enumerate}[(a)]
    \item This problem is a computation. Since $F_\eps$ is rotationally symmetric, we can use the given fact to see that 
    \[\Delta F_\eps = h''(|x|) + (n-2)|x|\inv h'(x)\]
    where 
    \[h(y) = c_n(y^2+\eps)^{(2-n)/2}.\]
    We compute that 
    \[h'(y) = yc_n(2-n)(y^2+\eps^2)^{-n/2}.\]
    Also, 
    \[h''(y) = c_n(2-n)(y^2+\eps^2)^{-n/2} - y^2c_n(2-n)n(y^2+\eps^2)^{-(n+2)/2}.\]
    So, plugging this in, we have 
    \begin{align*}
        \Delta F_\eps &= c_n (2-n)(|x|^2+\eps^2)^{-n/2}- |x|^2c_n(2-n)n(|x|^2+\eps^2)^{-(n-2)/2} - (n-1)(2-n)c_n(|x|^2 + \eps^2)^{-n/2}\\
        &=\eps^2 c_n(2-n)(|x|^2 + \eps^2)^{-(n+2)/2}\\
        &=-\eps^{-n} c_n(n-2)(|x/\eps|^2 + 1)^{-(n+2)/2}\\
        &= \eps^{-n}g(x/\eps),
    \end{align*}
    as we wanted. \qed
    \item Since $F_\eps$ is bounded by $F$, for any test function $\phi$, tho dominated convergence theorem gives us
    \begin{align*}
       \lim_{\eps \to 0} |(F-F_\eps)(\phi)| &= \lim_{\eps \to 0} \abs{\int c_n\del{|x|^{2-n} - (|x|2 + \eps^2)^{(2-n)/2} }\phi(x) dx} \\
       &=  \abs{\int c_n\del{ \lim_{\eps \to 0}\del{|x|^{2-n} - (|x|2 + \eps^2)^{(2-n)/2} }}\phi(x) dx}\\
       &=  \abs{\int c_n\del{0}\phi(x) dx}\\
       &= 0.
    \end{align*}
    So, $F_\eps \to F$ in the sense of distributions indeed. \qed
\end{enumerate}

\newpage
\problem{6} 
 \tri
\hop 
\solution
\begin{enumerate}[(a)]
    \item From problem 5(b), and what we did in class, we see since $F$ is compactly supported,$a \over b$, 
    \[\Delta u = \Delta F * f = \delta * f  = f.\]
    \qed
    \item We have seen that the Fourier transform of $\delta$ is 1, but let us confirm this. We see that for any test function $\phi$, 
    \begin{align*}
        \F(\delta)(\phi) &= \delta(\F(\phi)) \\
        &= \delta\del{\int \phi(x)e^{-ix\cdot y}dx}\\
        &= \int \phi(x)dx.
    \end{align*}
    So, by definition, $\F(\delta) = 1$.
    \hop 
    Now, note that using our Fourier transform of derivative rules, we have 
    \[\F(\Delta F) = \sum_{j=1}^n (-iy_j)^2 \F(F) = -|y|^2\F(F).\]
    Since $\F(\Delta F) = \F(\delta) = 1$, we can say 
    \[\F(F) = -\frac{1}{|y|^2}.\]
    \qed
\end{enumerate}

\end{document}