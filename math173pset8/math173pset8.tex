
\documentclass{article}
\usepackage{amsmath, amsfonts, amssymb, amsthm} %packages for math-related
\usepackage[margin=1.2in]{geometry}
\usepackage[shortlabels]{enumitem}
\usepackage[utf8]{inputenc}
\usepackage{graphicx} % package for inserting images
\usepackage{commath} % package for things like \del, \cbr, and \sbr. These handle parentheses well.
\usepackage[mathscr]{euscript}%for \scr command 
\usepackage{../commands} %package with all of the commands for this class
\usepackage{url}
\setlength{\parindent}{0em} % so paragraphs aren't indented
\newcommand{\lcm}{\text{lcm}}
\newcommand{\Hom}{\text{Hom}}
\newcommand{\Ann}{\text{Ann}}
\newcommand{\Cc}{{C_c^\infty(\RR)}}
% ********************************************************** %
%               THINGS TO EDIT BELOW THIS LINE               %
% ********************************************************** %
\newcommand{\D}{\nabla}
\renewcommand{\d}{\partial}
\usepackage{wrapfig}
\title{\textsc{MATH 173 Problem Set 7}}
\author{Stepan (Styopa) Zharkov}
\date{May 18, 2022}
\begin{document}
\maketitle
\problem{1} 
 \tri
\hop 
\solution
This problem is straightforward. Since $\ol{\Omega}$ is closed, and $c > 0$, there exists constants $C_1, C_2 >0$ such that $C_1 < c(x) < C_2$. Assume $C_1 < 1$ and $C_2 > 1$. If not, we can always choose smaller $C_1$ and larger $C_2$. So, for any $u \in C^1(\ol{\Omega})$, 
\begin{align*}
    ||u||^2_{H^1_c(\Omega)} &= \int_\Omega |u(x)|^2 dx + \int_\Omega c(x) |\D u (x)|^2 dx \\
    &\ge C_1^2\int_\Omega |u(x)|^2 dx + \int_\Omega C_1^2|\D u (x)|^2 dx\\
    &= C_1^2||u||^2_{H^1(\Omega)}.
\end{align*}
Similarly, 
\begin{align*}
    ||u||^2_{H^1_c(\Omega)} &= \int_\Omega |u(x)|^2 dx + \int_\Omega c(x) |\D u (x)|^2 dx \\
    &\le C_2^2\int_\Omega |u(x)|^2 dx + \int_\Omega C_2^2|\D u (x)|^2 dx\\
    &= C_2^2||u||^2_{H^1(\Omega)}.
\end{align*}
So, the statement follows for all  $u \in C^1(\ol{\Omega})$. By continuity and density, it follows that the statement holds for $u \in H^1(\Omega)$. \qed
\newpage
\problem{2} 
 \tri
\hop 
\solution
\begin{enumerate}
    \item Suppose $v = u+w = u' + w'$ where $u,u' \in M$, $w, w' \in M^\perp$. Then, we see 
    \[u' - u = u'+w - v = w - w'.\]
    But $u'-u \in M$ and $w - w' \in M^\perp$ and $M \cap M^\perp = \{0\}$. So, $u' - u = w-w' = 0$. Thus, the decomposition is unique. 
    \hop 
    We see that $u = u+0$. By uniqueness, $P(v) = u = P(u) = P(P(v))$, so $P = P^2$. \qed
    \item Let $v = u+w$ and $v' = u'+w'$ with $u,u' \in M$, $w, w' \in M^\perp$. We see 
    \begin{align*}
        \ang{Pv, v'}  &= \ang{u, l'+w'} = \ang{u, u'} + \ang{u,w'} = \ang{u, u'} = \ang{u,u'}+\ang{w, u'} = \ang{u+w, u'} = \ang{v, Pv'}
    \end{align*}
    \item TODO
\end{enumerate}

\newpage
\problem{3} 
 \tri
\hop 
\solution
TODO

\newpage
\problem{4} 
 \tri
\hop 
\solution
TODO

\newpage
\problem{5} 
 \tri
\hop 
\solution
TODO

\newpage
\problem{6} 
 \tri
\hop 
\solution
TODO

\end{document}