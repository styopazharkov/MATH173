
\documentclass{article}
\usepackage{amsmath, amsfonts, amssymb, amsthm} %packages for math-related
\usepackage[margin=1.2in]{geometry}
\usepackage[shortlabels]{enumitem}
\usepackage[utf8]{inputenc}
\usepackage{graphicx} % package for inserting images
\usepackage{commath} % package for things like \del, \cbr, and \sbr. These handle parentheses well.
\usepackage[mathscr]{euscript}%for \scr command 
\usepackage{../commands} %package with all of the commands for this class
\usepackage{url}
\setlength{\parindent}{0em} % so paragraphs aren't indented
\newcommand{\lcm}{\text{lcm}}
\newcommand{\Hom}{\text{Hom}}
\newcommand{\Ann}{\text{Ann}}

% ********************************************************** %
%               THINGS TO EDIT BELOW THIS LINE               %
% ********************************************************** %
\newcommand{\D}{\nabla}
\renewcommand{\d}{\delta}

\title{\textsc{MATH 173 Problem Set 1}}
\author{Stepan (Styopa) Zharkov}
\date{April 6, 2022}
\begin{document}
\maketitle
\problem{1} TODO \tri
\hop
\solution
First, we see that 
\[\D \times F = \D \times \D f = \del{\d_2\d_3 f - \d_3\d_2 f,\d_3\d_1 f - \d_3\d_1 f, \d_1\d_2 f - \d_1\d_2 f} = (0,0,0)\]
because order of differentiation does not matter. Also, 
\[\D \cdot F = \D \cdot \D f = \d_1\d_1 f +  \d_2\d_2 f  + \d_3\d_3 f  = \Delta f,\]
as we wanted. \qed


\newpage
\problem{2} TODO \tri
\hop
\solution 
\begin{enumerate}[(a)]
    \item Consider $u(x_1, x_2) = x_1 - x_2$. We see that $\d_1u+\d_2u=1-1=0$ and $u(x,x)=0$, but $u$ is nonzero.\qed
    \item Suppose $u(\hat x_1, \hat x_2) \ne 0$ for some $\hat x_1, \hat x_2$. Consider the function $f(s) = u(\hat x_1 + s, \hat x_2 + s)$. We see that $f(0)\ne 0$ and $f((-\hat x_1 -\hat x_2)/2)= u((\hat x_1-\hat x_2)/2, (-\hat x_1+\hat x_2)/2) = 0$. By the mean value theorem, there is some point where $f' \ne 0$. 
    \hop
    However, we see that $f'(s) = \d_1u(\hat x_1+s,\hat x_2+s)+\d_2u(\hat x_1+s, \hat x_2+s) = 0$. So, we have a contradiction and thus there is no such $\hat x_1, \hat x_2$ and $u = 0$. \qed
    \item Let $f_r(s) = u(r+s,-r+s)$. We $f'(s) = \d_1u(r+s,-r+s)+\d_2u(r+s, -r+s) = 0$, so $f_r$ is constant. Thus, $u(r,-r)$ defines all of $f_r$. Note that any point $(x_1, x_2)$ is expressed uniquely as $(r+s, -r+s)$, so the $f_r$ cover the entire plane with no overlap. 
    \hop
    In other words, Any solution can be described as $u(x_1, x_2) = g((x_1-x_2)/2)$ where $g(r): \RR \to \RR$ is a $C^1$ function. We also see that any choice of $g$ gives a solution, so this characterizes all solutions. \qed
\end{enumerate}

\newpage
\problem{3} TODO \tri
\hop
\solution
\begin{enumerate}[(a)]
    \item TODO
    \item TODO
\end{enumerate}


\newpage
\problem{4} TODO\tri
\hop
\solution We can rewrite the equation as 
\[x_1\d_1u+x_2\d_2u=(2-x_1)u\]
First, let's find the characteristic curves with starts on the curve $\Gamma: x_2 = e^{x_1}$. Any characteristic curve $f$ has $f_1'(s) = s$ and $f'_2(s) = s$ with $f_2(0)=e^{f_1(0)}$. 
\hop 
Solving this, we have $f_1(s) = re^s$ and $f_2(s) = e^re^s$ for some $r$. Let $f_r(s) = (re^s, e^re^s)$ be the characteristic curves, then. Since $e^s$ can be any positive number and the vector $(r,e^r)$ can point in any direction above the $x_1$-axis and above the line of slope $e$, we see that our characteristic curves cover the plane above these two lines.
\hop %TODO: image here. 
Let $y_r(s) = u(f_r(s))$. Since $f_r$ are characteristic curves, we know $y_r'(s)=(2-re^s)$ and $y_r(0)=re^0 = r$. Using our calculus methods, we have $dy/y = (2-re^s)ds$, so $\ln y = 22s-re^s+c$. The initial condition gives us that  
\[y_r(s)=e^{2s-re^s+r+\ln r}=re^{2s-re^s+r}.\]
Since we can express $x_1 = re^s$ and $x_2 = e^{r+s}$, we see that 
\[u(x_1,x_2) = y_r(s) = re^{2s-re^s+r} = re^s \cdot e^{s+r}\cdot e^{re^s} = x_1x_2e^{x_1}\]
for all points $(x_1, x_2)$ that are on some characteristic curve. So, we have found a solution $u$ that is uniquely determined on the region above the $x_1$-axis and the line with slope $e$. \qed

\newpage
\problem{5} TODO \tri
\hop
\solution
\begin{enumerate}[(a)]
    \item This problem is solved using the same idea as problem 4, but with three variables.
    \item TODO
\end{enumerate}

\newpage
\problem{6} TODO \tri
\hop
\solution TODO


\newpage
\problem{7} TODO \tri
\hop
\solution
\begin{enumerate}[(a)]
    \item TODO
    \item TODO
\end{enumerate}

\end{document}