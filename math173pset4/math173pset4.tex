
\documentclass{article}
\usepackage{amsmath, amsfonts, amssymb, amsthm} %packages for math-related
\usepackage[margin=1.2in]{geometry}
\usepackage[shortlabels]{enumitem}
\usepackage[utf8]{inputenc}
\usepackage{graphicx} % package for inserting images
\usepackage{commath} % package for things like \del, \cbr, and \sbr. These handle parentheses well.
\usepackage[mathscr]{euscript}%for \scr command 
\usepackage{../commands} %package with all of the commands for this class
\usepackage{url}
\setlength{\parindent}{0em} % so paragraphs aren't indented
\newcommand{\lcm}{\text{lcm}}
\newcommand{\Hom}{\text{Hom}}
\newcommand{\Ann}{\text{Ann}}
\newcommand{\Cc}{{C_c^\infty(\RR)}}
% ********************************************************** %
%               THINGS TO EDIT BELOW THIS LINE               %
% ********************************************************** %
\newcommand{\D}{\nabla}
\renewcommand{\d}{\partial}
\usepackage{wrapfig}
\title{\textsc{MATH 173 Problem Set 4}}
\author{Stepan (Styopa) Zharkov}
\date{April 27, 2022}
\begin{document}
\maketitle
\problem{1} TODO \tri
\hop
\solution
For this problem, we'll follow the first itinerary suggested by the hint. We know
\[f(x) = \int_0^x f'(t)dt.\]
Let $\chi_{[0,x]}$ be the characteristic function of $[0,x]$. We now see that by the Cauchy-Schwartz inequality,
\begin{align*}
    \int_0^1f(x)^2 dx &= \int_0^1\del{ \int_0^x f'(t)dt}dx \\
    &= \int_0^1\del{ \int_0^1 \chi_{[0,x]}f'(t)dt}^2dx \\
    & \le  \int_0^1\del{ \int_0^1 \chi_{[0,x]}^2dt}\del{ \int_0^1 f'(t)^2dt}dx \\
    &= \del{ \int_0^1 f'(t)^2dt} \int_0^1\del{ \int_0^1 \chi_{[0,x]}^2dt}dx \\
    &=  \del{ \int_0^1 f'(t)^2dt} \int_0^1 xdx \\
    &=\frac{1}{2}   \int_0^1 f'(t)^2dt.
\end{align*}
So, an absolute constant of $C = 1/2$ works. \qed

\newpage
\problem{2} TODO \tri
\hop
\solution
\begin{enumerate}[(a)]
    \item First, we see that for any $\eps > 0$, there exists an $R$ such that $|u(0,R)| \le \eps$. So, 
    \[\sup_{x \in \RR^n, t \in [0,t]}u(t,x) \ge -\eps\]
    and thus 
    \[\sup_{x \in \RR^n, t \in [0,t]}u(t,x) \ge 0.\]
    Also,
    \[\sup_{x \in \RR^n, t \in [0,t]}u(t,x) \ge \sup_{x \in \RR^n}u(0,x)\]
    so 
    \[\sup_{x \in \RR^n, t \in [0,t]}u(t,x) \ge \max\{0, \sup_{x \in \RR^n}u(0,x)\}.\]
    The other direction is more involved. Let
    \[C = \sup_{x \in \RR^n, t \in [0,t]}u(t,x).\]
    If $C = 0$, then we are done so let $C>0$.
    For any $\eps > 0$ such that $\eps < C$, let $x_0, t_0$ be such that $u(x_0, t_0) > C - \eps$. We know there exists an $R$ such that 
    \[\sup_{|x| > R, t \in [0,t]}u(t,x) < C - \eps.\]
    Let $R_0 > \max{t_0, R}$. We then see by the maximum principle on the hypercylinder with the $R_0$-bal as the base that $u$ achieves its supremum in the cylinder either on the base $\{(0,x): |x| \le R_0\}$ or on the wall $\{(x,t) : t \in [0,T], |x| = R_0\}$. However we saw that the supremum on the wall is less than $C-\eps < u(t_0, x_0)$, which is inside the cylinder. So, the supremum is attained on the base and there exits an $x$ such that $u(0,x) \ge u(t_0, x_0) \ge C- \eps$. Since $\eps$ was arbitrarily small, we have shown that RHS is at least $C$, so we can conclude that 
    \[\sup_{x \in \RR^n, t \in [0,t]}u(t,x) = \max\{0, \sup_{x \in \RR^n}u(0,x)\},\]
    as we wanted. \qed
    \item If $u$ and $u'$ are solutions that go to 0 at infinity uniformly, then consider $v = u - u'$. We see that $v(0,x) = 0$ and that $v \to 0$ at infinity uniformly as well, so by part (a), we know $v \le 0$. By a similar reasoning $-v = u' - u \le 0$ as well. So, $v = 0$ and thus $u = u'$. Thus, the solution in the given class of functions must be unique.\qed 
\end{enumerate}


\newpage
\problem{3} TODO \tri
\hop
\solution
\begin{enumerate}[(a)]
    \item This problem is just computation. By the product of sines formula, 
    \[\sin(n\pi x)\sin(n\pi x) = \frac{1}{2}(\cos((n-m)\pi x) - \cos((n+m)\pi x)).\]
    So, if $m \ne n$, then 
    \begin{align*}
        \int_0^1 \sin n\pi x \sin m \pi x dx&= \frac{1}{2}\sbr{\frac{1}{(n-m)\pi}\sin((n-m)\pi x)}_0^1+\frac{1}{2}\sbr{\frac{1}{(n+m)\pi}\sin((n+m)\pi x)}_0^1\\
        &= \frac{1}{2}(0+0)\\
        &= 0.
    \end{align*}
    For the other case, if $m = n$, then
    \begin{align*}
        \int_0^1 \sin n\pi x \sin m \pi x dx&=  \frac{1}{2}\int_0^1(\cos((0)\pi x) - \cos((2n)\pi x)) dx \\
        &= \frac{1}{2}\sbr{\cos(0) - \frac{1}{2n\pi}\cos(2n\pi x)}_0^1 \\
        &= \frac{1}{2}(1-0) \\
        &= \frac{1}{2}.
    \end{align*}
    So, we have shown the equality we wanted to show. \qed
    \item This problem can be solved by heavy computation and using part (a). Instead, we'll use what we have seen in class. Fix $s$ and $y$. We know 
    \[u(t,x)=\int_0^1K(t,x,r)K(s,r,y)dr\]
    gives a solution for $u_t = u_{xx}$ with initial conditions $u(0,x) = K(s,x,y)$. We also know that the heat kernel satisfies the heat equation for $t>0$. Note that this means
    \[u'(t,x) = K(t+s, x, y)\]
    satisfies $u'_t = u'_{yy}$. Also note that we can check that $u(0,x) = K(s,x,y)$, so the initial conditions are the same as above. Since the solution with the same initial conditions is unique, our two solutions must be the same, so 
    \[\int_0^1K(t,x,r)K(s,r,y)dr=K(t+s, x, y),\]
    as we wanted. \qed
\end{enumerate}


\newpage
\problem{4} TODO \tri
\hop
\solution
This problem is similar to the use of Poincar\'e's inequality in class, but with a positive-definite matrix in place of the $c^2$ constant.
\hop 
Let us define 
\[E= \int_\Omega (A(x)\D u \cdot \D u + qu^2)dx.\]
Notice that all parts in the integral are non-negative, so $E$ is also non-negative. By the same integration by parts that we've done in class, we see that the dirichlet boundary conditions give us
\begin{align*}
    E &= 0 - \int_\Omega (\D \cdot (A(x) \D u) - qu)u dx \\
    &= -\int_\Omega fu dx.
\end{align*}
Applying the Cauchy-Schwartz inequality, then the AMGM inequality, then the Poincar\'e inequality, we know there exists a $C > 0$ such that 
\begin{align*}
    E = |E| &= \abs{\int_\Omega fu dx} \\
    &\le \del{\int_\Omega f^2 dx}^{1/2} \del{\int_\Omega u^2 dx}^{1/2} \\
    &= \del{\frac{C}{c_o}\int_\Omega f^2 dx}^{1/2} \del{\frac{c_0}{C}\int_\Omega u^2 dx}^{1/2} \\
    &\le \frac{1}{2}\frac{C}{c_0}\int_\Omega f^2 dx + \frac{1}{2}\frac{c_0}{C}\int_\Omega u^2 dx \\ 
    & \le \frac{C}{2c_0}\int_\Omega f^2 dx + \frac{C}{2}\frac{c_0}{C}\int_\Omega \abs{\D u}^2 dx \\
    &=  \frac{C}{2c_0}\int_\Omega f^2 dx +\frac{c_0}{2}\int_\Omega \abs{\D u}^2 dx.
\end{align*}
We also know that 
\begin{align*}
    E&= \int_\Omega (A(x)\D u \cdot \D u + qu^2)dx\\
    &\ge \int_\Omega (A(x)\D u \cdot \D u)dx\\
    &\ge \int_\Omega (c_0\abs{\D u}^2 \cdot \D u)dx.
\end{align*}
Thus, 
\[ \int_\Omega (c_0\abs{\D u}^2 \cdot \D u)dx \le \frac{C}{2c_0}\int_\Omega f^2 dx +\frac{c_0}{2}\int_\Omega \abs{\D u}^2 dx,\]
so, subtracting, we get 
\[ \int_\Omega \frac{c_0}{2} \abs{\D u}^2 dx \le \frac{C}{2c_0}\int_\Omega f^2 dx.\]
Also, by Poincar\'e's inequality, 
\[\int_\Omega \frac{c_0}{2}u^2 dx \le C \int_\Omega\frac{c_0}{2} \abs{\D u}^2dx \le \frac{C^2}{2c_0}\int_\Omega f^2 dx.\]
Adding the previous two results together and dividing by $c_0/2$, we have 
\[\int_\Omega \del{\abs{\D u}^2  + u^2}dx \le \frac{C+C^2}{c_0^2}\int_\Omega f^2 dx.\]
So, a constant of $\frac{C+C^2}{c_0^2}$ works, and we are done. \qed


\newpage
\problem{5} TODO \tri
\hop
\solution
This problem is solved the same as our familiar quasilinear PDEs, but the system of ODEs is a little more annoying than usual. Let $f(s) = (x(s), y(s),z(s))$ be a characteristic curve. The initial conditions give us that $y(0) = 1$ and $z(0) = 1 + x(0)$. Let $r = x(0)$ be the variable parameter for the characteristic curve. 
\hop 
We see that 
\begin{align*}
    x' &= y + z \\
    y' &= y\\
    z' &= x - y.
\end{align*}
So, we can immediately deduce from the initial conditions that $y(s) = e^s$. We also see that $(x+z)' = x'+z' = x+z$, and we know $x(0)+z(0) = 2r +1$, so \[x+z = (2r+1)e^s.\] 
Plugging this back in, we have 
\begin{align*}
    & x' = y+z = e^s+(2r+1)e^s - x \\ 
    \implies & x' + x = e^s + (2r+1)e^s \\
    \implies & (xe^{s})' = e^s(x' + x) = e^{2s} + (2r+1)e^{2s}= 2(r+1)e^{2s} \\
    \implies & xe^s = (r+1)e^{2s} + C\\
\end{align*}
for some constant $C$. Since $x(0) = r$, we know $x(0)e^0 = r$, so $C = -1$. Thus, 
\begin{align*}
    x &= (r+1)e^s - e^{-s}\\
    z &= re^s + e^{-s}.
\end{align*}
So, we can see that for $y > 0$, we can pick $r$ such that 
\[u(x,y) = z(s) = x - y + \frac{2}{y}.\]
This is the solution we were after. \qed


\newpage
\problem{6} TODO \tri
\hop
\solution
We can factor the equation to see that 
\[(\d_x+8\d_y)(\d_x-2\d_y)u = 0.\]
So, we know the solution is of the form 
\[u(x,y) = f(8x-y) + g(2x+y).\]
Now, we can use the initial conditions to solve for $f$ and $g$. We know 
\begin{align*}
    f(-10x)+g(0) &= x\\
    f(8x) + g(2x) &= \sin x.
\end{align*}
So, 
\begin{align*}
    f(40x) + g(0) &= -4x \\
    g(10x) - g(0) &= 4x + \sin 5x.
\end{align*}
We can conclude that 
\begin{align*}
    u(x,y) &= f(8x - y) + g(2x + y) \\
    &= \frac{8x-y}{-10} - g(0) + 4\del{\frac{2x+y}{10}} + \sin \del{\frac{2x+y}{2}} + g(0)\\
    &= \frac{y}{2} + \sin\del{x+ \frac{y}{2}}
\end{align*}
is our solution. \qed


\newpage
\problem{7} TODO \tri
\hop
\solution
\begin{enumerate}
    \item For this problem, we can follow the standard uniqueness proof using the maximum principle. Suppose $u$ and $u'$ are two solutions to the equation. $v = u - u'$. We see that $v$ solves $v_t - v_{xx} = 0$ and $v(x,0) = 0$ and $v(0,t) = v(1,t) = 0$. Thus, by the maximum principle on inhomogenous heat equations, $v \le 0$. We see that $-v = u' - u$ solves the same equation with the same properties, so $-v \le 0$. Thus, $v= 0$. This means that $ u = u'$ and the solution is unique. \qed
    \item  We know that the solution has the form 
    \item \begin{align*}
        u(t,x) &= \int_0^1K(t,x,y)f(y)dy + \int_0^t\int_0^1 K(t-s,x,y)e^{-t}\sin 3\pi y dy ds \\
        &= \int_0^1K(t,x,y)y\sin \pi x dy + \int_0^t\int_0^1 K(t-s,x,y)e^{-t}\sin 3\pi y dy ds
    \end{align*}
    where 
    \begin{align*}
        K(t,x,y) = 2 \sum_{n=1}^\infty e^{-(\pi n)^2 t}\sin(n \pi x) \sin(n \pi y)
    \end{align*}
    is the heat kernel. \qed
\end{enumerate}

\end{document}