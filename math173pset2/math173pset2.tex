
\documentclass{article}
\usepackage{amsmath, amsfonts, amssymb, amsthm} %packages for math-related
\usepackage[margin=1.2in]{geometry}
\usepackage[shortlabels]{enumitem}
\usepackage[utf8]{inputenc}
\usepackage{graphicx} % package for inserting images
\usepackage{commath} % package for things like \del, \cbr, and \sbr. These handle parentheses well.
\usepackage[mathscr]{euscript}%for \scr command 
\usepackage{../commands} %package with all of the commands for this class
\usepackage{url}
\setlength{\parindent}{0em} % so paragraphs aren't indented
\newcommand{\lcm}{\text{lcm}}
\newcommand{\Hom}{\text{Hom}}
\newcommand{\Ann}{\text{Ann}}

% ********************************************************** %
%               THINGS TO EDIT BELOW THIS LINE               %
% ********************************************************** %
\newcommand{\D}{\nabla}
\renewcommand{\d}{\partial}

\title{\textsc{MATH 173 Problem Set 2}}
\author{Stepan (Styopa) Zharkov}
\date{April 13, 2022}
\begin{document}
\maketitle
\problem{1} TODO
\hop
\solution
This is a semilinear PDE and can be solved with the same methods as last week. Let 
Let $(x(s), y(s))$ be a characteristic curve. We see that $x'(s) = 1, y'(s) = \cos$ with $x(0) = 0$. 
\hop
Solving this, we get that $x(s) = s, y(s) = \sin(s) + a$. Let $f_a(s) = (s, 
sin(s)+a)$ be the characteristic curve. Also, let $\omega_a(s) = u(f_a(s))$.  We know $\omega_a'(s) = y(s) = \sin(s) + a$ and $\omega_a(0) = a$. So, we see that $\omega_a(s) = -\cos(s)+as +a +1$. We see that $x= s$, $y= \sin(s)+a$. Solving for $a, s$, we have that $s =x$ and $a = y - \sin(x) + y$
\qed


\newpage
\problem{2} TODO
\hop
\solution
\begin{enumerate}
    \item TODO
    \item TODO
\end{enumerate}


\newpage
\problem{3} TODO
\hop
\solution
TODO \qed


\newpage
\problem{4} TODO
\hop
\solution
\begin{enumerate}
    \item TODO
    \item TODO
\end{enumerate}


\newpage
\problem{5} TODO
\hop
\solution
\begin{enumerate}
    \item TODO
    \item TODO
\end{enumerate}


\newpage
\problem{6} TODO
\hop
\solution

TODO 


\newpage
\problem{7} TODO
\hop
\solution
\begin{enumerate}
    \item TODO
    \item TODO
\end{enumerate}
\end{document}