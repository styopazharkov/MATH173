
\documentclass{article}
\usepackage{amsmath, amsfonts, amssymb, amsthm} %packages for math-related
\usepackage[margin=1.2in]{geometry}
\usepackage[shortlabels]{enumitem}
\usepackage[utf8]{inputenc}
\usepackage{graphicx} % package for inserting images
\usepackage{commath} % package for things like \del, \cbr, and \sbr. These handle parentheses well.
\usepackage[mathscr]{euscript}%for \scr command 
\usepackage{../commands} %package with all of the commands for this class
\usepackage{url}
\setlength{\parindent}{0em} % so paragraphs aren't indented
\newcommand{\lcm}{\text{lcm}}
\newcommand{\Hom}{\text{Hom}}
\newcommand{\Ann}{\text{Ann}}
\newcommand{\Cc}{{C_c^\infty(\RR)}}
% ********************************************************** %
%               THINGS TO EDIT BELOW THIS LINE               %
% ********************************************************** %
\newcommand{\D}{\nabla}
\renewcommand{\d}{\partial}
\usepackage{wrapfig}
\title{\textsc{MATH 173 Problem Set 4}}
\author{Stepan (Styopa) Zharkov}
\date{April 27, 2022}
\begin{document}
\maketitle
\problem{1} Suppose that $f : \RR \to \CC$ is continuous with $(1 + |x|)^Nf(x)$ bounded for some $N > 1$ (or
indeed simply that $f \in L^1(\RR))$ and $a \in R$:
\begin{enumerate}[(a)]
    \item Let $f_a(x) = f(x - a)$. Show that $\hat{f}_a(y) = e^{-iay} \hat{f}(y)$.
\item Let $g_a(x) = e^{ixa}f(x)$. Show that $\hat{g}_a(y) = \hat{f}(y - a)$. 
\end{enumerate}
 \tri
\hop 
\solution
\begin{enumerate}[(a)]
    \item This problem is a simple computation. We see that, with a change of variables $z = x-a$, we have
    \begin{align*}
        \hat{f_a}(y) &= \cal{F}(f(x-a))(y) \\
        &= \int e^{-ixy}f(x-a)dx \\
        &= \int e^{-i(z+a)y}f(z)dx \\
        &= e^{iay}\int e^{-izy}f(z)dx \\
        &= e^{iay}\hat{f}(y),
    \end{align*}
    as we wanted. \qed
    \item This problem is even simpler computation. We see that 
    \begin{align*}
        \hat{g_a}(y) &= \cal{F}(e^{ixa}f(x))\\
        &= \int e^{-ixy}e^{ixa}f(x)dx \\
        &= \int e^{-ix(y-a)}f(x)dx \\
        &= \hat{f}(y-a),
    \end{align*}
    as we wanted. \qed
\end{enumerate}


\newpage
\problem{2} Compute the Fourier transform of the following functions:
\begin{enumerate}[(a)]
    \item $\chi_{(-a,a)}$
    \item $f(x) = e^{a|x|}, a > 0$
    \item $g(x) = |x|^ne^{-a|x|}, a>0, n \in \NN$. \tri
\end{enumerate} 
\solution
\begin{enumerate}[(a)]
    \item This problem is also computation.
    \begin{align*}
        (\cal{F}\chi_{(-a,a)}(y)) &= \int_{-\infty}^\infty e^{-ixy}\chi_{(-a,a)}(x) dx \\
        &= \int_{-a}^a e^{-ixy}dx \\
        &=\begin{cases}
            -\frac{i}{y}\del{e^{-iay}-e^{iay}} &\text{ if } y \ne 0\\
            2a &\text{ if } y =0
        \end{cases}
    \end{align*}
    \qed
    \item  Note that since $y \in \RR$ and $a > 0$, we know $iy - a \ne 0$ and $iy+a \ne 0$ so we can divide by them. So, 
    \begin{align*}
        (\cal{F}(e^{-a|x|})(y) &= \int_{-\infty}^{\infty}e^{-ixy}e^{a|x|}dx\\
        &= \int_{0}^{\infty}e^{-x(iy+a)}dx+\int_{-\infty}^{0}e^{-x(iy-a)}dx \\
        &= \sbr{-\frac{1}{iy+a}e^{-x(iy+a)}}_0^\infty+\sbr{-\frac{1}{iy-a}e^{-x(iy-a)}}_{-\infty}^0\\
        &= \frac{1}{iy+a} - \frac{1}{iy-a}
    \end{align*}
    because $a>0$. \qed
    \item In this problem, we will use repeated integration by parts. We see that executing integration by parts, we have
    \begin{align*}
        \cal{F}(|x|^ne^{-a|x|})(y) &= \int_{-\infty}^\infty |x|^ne^{-ixy-a|x|} dx\\
        &=  \int_{-\infty}^0 (-x)^ne^{-ixy+ax} dx+   \int_0^\infty x^ne^{-ixy-ax}dx\\
        &=  \int_{-\infty}^0 (-x)^ne^{-ix(y-a)} dx+   \int_0^\infty x^ne^{-ix(y+a)}dx \\
        &=  \frac{-n}{yi-a}\int_{-\infty}^0 (-x)^{n-1}e^{-ix(y-a)} dx+  \frac{n}{yi+a} \int_0^\infty x^{n-1}e^{-ix(y+a)}dx.
    \end{align*}
    Note that the boundary terms vanish in the integration by parts. Repeating integration by parts $n$ times, we see that
    \begin{align*}
        \cal{F}(|x|^ne^{-a|x|})(y) &= (-1)^n\frac{n!}{(yi-a)^n}\int_{-\infty}^0 e^{-ix(y-a)} dx+  \frac{n!}{(yi+a)^n} \int_0^\infty e^{-ix(y+a)}dx\\
        &=  (-1)^{n+1}\frac{n!}{(yi-a)^{n+1}}+  \frac{n!}{(yi+a)^{n+1}},
    \end{align*}
    and $yi\pm a$ does not vanish because $a>0$. \qed
\end{enumerate}



\newpage
\problem{3} 
\begin{enumerate}[(a)]
    \item Show that if $f$ is even (or odd) then so is its Fourier transform.
    \item Suppose that $f$ is even and real valued. Show that $\hat{f}$ is also real valued. What can you say about
    $\hat{f}$ when $f$ is odd and real valued.
\end{enumerate}
\tri
\hop 
\solution
\begin{enumerate}[(a)]
    \item The solution to this is straightforward. As Tadashi Tokieda would say, ``follow your nose''. First, let $f(x)= f(-x)$. Then, letting $z=-x$ power a change of variables, we see that 
    \begin{align*}
        \hat{f}(y) &= \int_{-\infty}^\infty e^{-ixy}f(x)dx\\
        &=  \int_{\infty}^{-\infty} -e^{izy}f(-z)dz\\
        &=  \int_{-\infty}^{\infty} e^{izy}f(-z)dz\\
        &=  \int_{-\infty}^{\infty} e^{-iz(-y)}f(z)dz\\
        &= \hat{f}(-y).
    \end{align*}
    Similarly, now let $f(x)= -f(-x)$. Then, 
    \begin{align*}
        \hat{f}(y) &= \int_{-\infty}^\infty e^{-ixy}f(x)dx\\
        &=  \int_{\infty}^{-\infty} -e^{izy}f(-z)dz\\
        &=  \int_{-\infty}^{\infty} e^{izy}f(-z)dz\\
        &=  \int_{-\infty}^{\infty}- e^{-iz(-y)}f(z)dz\\
        &= -\hat{f}(-y).
    \end{align*}
    So, the fourier transform preserves evenness and oddness. \qed
    \item Let $f$ be even. Then, 
    \begin{align*}
        \hat{f}(y) &= \int_{-\infty}^\infty e^{-ixy}f(x)dx \\
        &= \int_{-\infty}^0 e^{-ixy}f(x)dx + \int_{0}^\infty e^{-ixy}f(x)dx\\
        &= \int_{0}^\infty e^{ixy}f(-x)dx + \int_{0}^\infty e^{-ixy}f(x)dx\\
        &= \int_{0}^\infty e^{ixy}f(x)dx + \int_{0}^\infty e^{-ixy}f(x)dx\\
        &= \int_{0}^\infty\del{ e^{ixy}+ e^{-ixy}}f(x)dx\\
    \end{align*}
    This is real valued because $ e^{ixy}+ e^{-ixy} \in \RR$ and $f(x) \in \RR$. 
    \hop 
    The idea is the same when $f$ is odd. By a similar computation,
    \begin{align*}
        \hat{f}(y) &= \int_{-\infty}^\infty e^{-ixy}f(x)dx \\
        &= \int_{-\infty}^0 e^{-ixy}f(x)dx + \int_{0}^\infty e^{-ixy}f(x)dx\\
        &= \int_{0}^\infty e^{ixy}f(-x)dx + \int_{0}^\infty e^{-ixy}f(x)dx\\
        &= \int_{0}^\infty -e^{ixy}f(x)dx + \int_{0}^\infty e^{-ixy}f(x)dx\\
        &= \int_{0}^\infty\del{ -e^{ixy}+ e^{-ixy}}f(x)dx\\
    \end{align*}
    We see that this integral has no real part because $-e^{ixy}+ e^{-ixy}$ has no real part and $f(x)\in \RR$. \qed
    %TODO: cartoon of adding two vectors
\end{enumerate}


\newpage
\problem{4} Define the Hermite functions by $h_n = (x - d
dx
)^n e^{-x^2/2}$, such that $h_0 = e^{-x^2/2}$ and
$h_{n+1}(x) = xh_n(x) - h'_n(x)$. Show that $\hat{h}_n = (-i)^n\sqrt{2\pi} h_n$. \tri
\hop 
\solution
This proof follows by induction. From class we know that 
\[\hat{h_0}(y)=\sqrt{2\pi}e^{-y^2/2} = \sqrt{2\pi}h_0(y).\] 
So, the base case holds. Now, suppose the statement is true up to $h_n$. Then, we see that by linearity of $\cal{F}$ and the properties we have seen in class,
\begin{align*}
    \cal{F}(h_{n+1}) &= \cal{F}(xh_n)-\cal{F}(h_n')\\
    &= i(\cal{F}h_n)' - iy(\cal{F}h_n) \\
    &= -i(y\sqrt{2\pi}(-i)^n h_n - (\sqrt{2\pi}(-i)^n h_n)') \\
    &= (-i)^{n+1}\sqrt{2\pi}(yh_n - h_n')\\
    &= (-i)^{n+1}\sqrt{2\pi}h_{n+1}.
\end{align*}
So, by induction, the statement holds for all $n$. \qed

\newpage
\problem{5} Prove the Poisson summation formula
\[\sum_
{n\in\ZZ}
f(n)e^{-i\pi n} = \sum_
{m\in \ZZ}\hat{f}(x + 2\pi m),\]
where $f$ satisfies $|f(x)|+ |\hat{f}(x)|\le C(1 + |x|)^{-a}$ for some $a > 1$.  \tri
\hop 
\solution
We will follow the hint. Let us compute the Fourier coefficients of the RHS. We get that with a change of variables,
\begin{align*}
    c_n  &= \frac{1}{2\pi} \int_{-\pi}^\pi e^{inx}\sum_{m \in \ZZ}\hat{f}(x+2\pi m) dx \\
    &= \frac{1}{2\pi} \sum_{m \in \ZZ} \int_{-\pi}^\pi e^{inx}\hat{f}(x+2\pi m) dx \\
    &= \frac{1}{2\pi} \sum_{m \in \ZZ} \int_{-\pi+2\pi m}^{\pi+2\pi m} e^{in(z - 2\pi m)}\hat{f}(z) dz \\
    &= \frac{1}{2\pi} \sum_{m \in \ZZ} \int_{-\pi+2\pi m}^{\pi+2\pi m} e^{inz}\hat{f}(z) dz \\
    &= \frac{1}{2\pi} \int e^{inz}\hat{f}(z) dz \\
    &= \cal{F}\inv(\hat{f})(n) \\
    &= f(n).
\end{align*}
So, assuming that the Fourier series converges to the function, 
\begin{align*}
    \sum_{m \in \ZZ}\hat{f}(x+2\pi m) &=  \sum_{n \in \ZZ}c_ne^{-inx}\\
    &= \sum_{n \in \ZZ}f(n)e^{-inx},
\end{align*}
which is what we wanted to prove. \qed
%TODO: fourier people cartoon. 


\newpage
\problem{6} Prove the Whittaker–Shannon interpolation formula
\[f(x) = \sum_{n\in \ZZ}
f(n)\text{sinc}(\pi(x -n))\]
provided $\supp(\hat{f}) \sse [-\pi,\pi]$ and $|f(x)|\le C(1 + |x|)^{-a}$. Here $\text{sinc}(x) = \sin(x)/x$.  \tri
\hop 
\solution
Since $\supp(\hat{f}) \sse [-\pi, \pi]$, we see that for $x \in (-\pi,\pi)$, by problem 5,
\begin{align*}
    \sum_{n \in \ZZ}f(n) e^{-inx} = \sum_{m \in \ZZ} \hat{f}(x + 2\pi m) = \hat{f}(x).
\end{align*}
Note that we saw that a finite number of points do not change the fourier transform, so the points $-\pi, \pi$ are not a problem. Assuming that the inverse of the Fourier transform reverts the action of the Fourier transform, as in problem 5, we see that 
\begin{align*}
    f(y) &= \cal{F}\inv(\hat{f})(y) \\
    &= \frac{1}{2\pi}\int e^{inx}\sum_{n\in \ZZ} f(n)e^{-ixy}dx \\
    &= \frac{1}{2\pi}\int_{-\pi}^\pi e^{ixy}\sum_{n\in \ZZ} f(n)e^{-inx}dx \\
    &=\sum_{n\in \ZZ} f(n) \frac{1}{2\pi}\int_{-\pi}^\pi e^{ixy}e^{-inx}dx \\
    &=\sum_{n\in \ZZ} f(n) \frac{1}{2\pi}\int_{-\pi}^\pi e^{ix(y-n)}dx \\
    &=\sum_{n\in \ZZ} f(n) \frac{1}{2\pi}\sbr{\frac{1}{i(y-n)}e^{ix(y-n)}}_{-\pi}^\pi dx \\
    &=\sum_{n\in \ZZ} f(n) \del{\frac{e^{i\pi(y-n)}-e^{-i\pi(y-n)}}{2\pi i(y-n)}} dx\\
    &=\sum_{n\in \ZZ} f(n) \del{\frac{\sin(\pi(y-n))}{\pi (y-n)}} dx\\
    &=\sum_{n\in \ZZ} f(n) \text{sinc}(\pi(y-n)),\\
\end{align*}
exactly as we wanted.\qed



\newpage
\problem{7} 
\begin{enumerate}
    \item Use the Fourier transform to solve the following free Schr\:odinger equation $iut = -uxx$,
    with the initial condition $u(0,x) = g(x)$, where $u : \RR\times \RR \to \CC$ and $g \in \cal{S}(\RR)$ is a given Schwarz function.
    You may leave your solution as the inverse Fourier transform of a Schwartz function (you do not need
    to evaluate it explicitly).
    \item  Evaluate the solution in part (a) explicitly if $g(x) = 1\sqrt{2\pi} e^{-x^2/
    2}$, and show that $\int \RR|u(x,t)|^2dx = 1$ for
    all $t$.
\end{enumerate}
\tri
\hop 
\solution
TODO
\end{document}