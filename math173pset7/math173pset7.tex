
\documentclass{article}
\usepackage{amsmath, amsfonts, amssymb, amsthm} %packages for math-related
\usepackage[margin=1.2in]{geometry}
\usepackage[shortlabels]{enumitem}
\usepackage[utf8]{inputenc}
\usepackage{graphicx} % package for inserting images
\usepackage{commath} % package for things like \del, \cbr, and \sbr. These handle parentheses well.
\usepackage[mathscr]{euscript}%for \scr command 
\usepackage{../commands} %package with all of the commands for this class
\usepackage{url}
\setlength{\parindent}{0em} % so paragraphs aren't indented
\newcommand{\lcm}{\text{lcm}}
\newcommand{\Hom}{\text{Hom}}
\newcommand{\Ann}{\text{Ann}}
\newcommand{\Cc}{{C_c^\infty(\RR)}}
% ********************************************************** %
%               THINGS TO EDIT BELOW THIS LINE               %
% ********************************************************** %
\newcommand{\D}{\nabla}
\renewcommand{\d}{\partial}
\usepackage{wrapfig}
\title{\textsc{MATH 173 Problem Set 7}}
\author{Stepan (Styopa) Zharkov}
\date{May 18, 2022}
\begin{document}
\maketitle
\problem{1} TODO
 \tri
\hop 
\solution
For a harmonic function, we know the average over the surface of any ball is equal to the value at the center. We also know that the average over the inside of the ball is equal to the average in the center because the average on the inside is simply an integral of the averages over many concentric balls. 

TODO: img 

Now, consider $B_x := B(x,R)$ and $B_y := B(y, R+d)$ where $d = |x - y|$, as the hint suggests. Then, we see that $B_x \sse B_y$. 

TODO: img 

Let $\text{Avg}(\Omega)$ denote the average of $u$ over the region $\Omega$. Let $|\Omega|$ denote the area of $\Omega$. We see that 
\newcommand{\Avg}{\text{Avg}}
\begin{align*}
    \Avg(B_y) &= \Avg(B_x)\frac{|B_x|}{|B_y|} + \Avg(B_y\setminus B_x)\frac{|B_y \setminus B_x|}{|B_y|} \\
    & \ge \Avg(B_x)\frac{|B_x|}{|B_y|} + \Avg(B_y\setminus B_x)
\end{align*}


\newpage
\problem{1} TODO
 \tri
\hop 
\solution
TODO


\newpage
\problem{1} TODO
 \tri
\hop 
\solution
TODO


\newpage
\problem{1} TODO
 \tri
\hop 
\solution
TODO


\newpage
\problem{1} TODO
 \tri
\hop 
\solution
TODO


\newpage
\problem{1} TODO
 \tri
\hop 
\solution
TODO


\newpage
\problem{1} TODO
 \tri
\hop 
\solution
TODO


\end{document}