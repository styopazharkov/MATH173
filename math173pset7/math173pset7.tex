
\documentclass{article}
\usepackage{amsmath, amsfonts, amssymb, amsthm} %packages for math-related
\usepackage[margin=1.2in]{geometry}
\usepackage[shortlabels]{enumitem}
\usepackage[utf8]{inputenc}
\usepackage{graphicx} % package for inserting images
\usepackage{commath} % package for things like \del, \cbr, and \sbr. These handle parentheses well.
\usepackage[mathscr]{euscript}%for \scr command 
\usepackage{../commands} %package with all of the commands for this class
\usepackage{url}
\setlength{\parindent}{0em} % so paragraphs aren't indented
\newcommand{\lcm}{\text{lcm}}
\newcommand{\Hom}{\text{Hom}}
\newcommand{\Ann}{\text{Ann}}
\newcommand{\Cc}{{C_c^\infty(\RR)}}
% ********************************************************** %
%               THINGS TO EDIT BELOW THIS LINE               %
% ********************************************************** %
\newcommand{\D}{\nabla}
\renewcommand{\d}{\partial}
\usepackage{wrapfig}
\title{\textsc{MATH 173 Problem Set 7}}
\author{Stepan (Styopa) Zharkov}
\date{May 18, 2022}
\begin{document}
\maketitle
\problem{1} TODO
 \tri
\hop 
\solution
For a harmonic function, we know the average over the surface of any ball is equal to the value at the center. We also know that the average over the inside of the ball is equal to the average in the center because the average on the inside is simply an integral of the averages over many concentric balls. 

TODO: img 

Let $x$ and $y$ be two arbitrary points. Now, consider $B_x := B(x,R)$ and $B_y := B(y, R+d)$ where $d = |x - y|$, as the hint suggests. Then, we see that $B_x \sse B_y$. 

TODO: img 

Let $\text{Avg}(\Omega)$ denote the average of $u$ over the region $\Omega$. Let $|\Omega|$ denote the area of $\Omega$. We see that 
\newcommand{\Avg}{\text{Avg}}
\begin{align*}
    \Avg(B_y) &= \Avg(B_x)\frac{|B_x|}{|B_y|} + \Avg(B_y\setminus B_x)\frac{|B_y \setminus B_x|}{|B_y|} \\
    & \ge \Avg(B_x)\frac{|B_x|}{|B_y|}.
\end{align*}
In the second step, we used that $u$ is non-negative. Note that $\lim_{R \to \infty} \frac{|B_x|}{|B_y|} = 1$, so for any $\eps$, we can find $R$ large enough that
\[u(y) = \Avg(B_y) \ge  \Avg(B_x)\frac{|B_x|}{|B_y|} \ge \Avg(B_x)(1-\eps) = u(x)(1-\eps).\] 
Thus, we can conclude that $u(y) \ge u(x)$. By an identical argument, we also know $u(y) \le u(x)$. Since $x$ and $y$ were arbitrary, $u$ must be the constant function. \qed 


\newpage
\problem{2} TODO
 \tri
\hop 
\solution
\begin{enumerate}[(a)]
    \item By the linearity properties of $\ang{\cdot, \cdot}$, we know 
    \begin{align*}
        ||v+w||^2 + ||v - w||^2 &= \ang{v+w, v+w} + \ang{v-w, v-w} \\
        &= \ang{v,v} + \ang{v,w} + \ang{w,v} + \ang{w,w} +\ang{v,v}-\ang{v,w} - \ang{w,v} + \ang{w,w} \\
        &= 2\ang{v,v} + 2 \ang{w,w}\\
        &= 2(||v||^2 + ||w||^2)
    \end{align*}
    for any $v,w$. \qed
    \item Let $\Omega_1, \Omega_2 \sse \Omega$ be nonempty oben balls where $\Omega_1 \cap \Omega_2 = \varnothing$. Let $f$ be a bump function such that $\supp{f} \sse \Omega_1$ and $f \ge 0$ and $f$ achieves its maximum of 1 in $\Omega_1$. Similarly, let $g$ be a bump function such that $\supp{g} \sse \Omega_2$ and $g \ge 0$ and $g$ achieves its maximum of 1 in $\Omega_2$.
    \hop 
    We see that 
    \[||f-g||^2 + ||f+g||^2 = 1+1 \ne 2(1+1) = 2(||f||^2 + ||g||^2).\]
    So the parallelogram law does not hold. By part (a), this means this norm cannot be induced by an inner product. \qed
\end{enumerate}


\newpage
\problem{3} TODO
 \tri
\hop 
\solution
Let $\ang{f,g} = \int_{-\pi}^\pi f\bar{g}$. We know from class that this is an inner product and we see that it induces the given norm. We saw that the functions $1, \cos(x), \sin(x), \cos(2x), \sin(2x)$ are orthogonal. So, by the best approximation lemma, we can take the inner products to find the coefficients. This gives 
\begin{align*}
    a_0 &= \frac{\ang{|x|, 1}}{\ang{1,1}} = \frac{\pi}{2}\\
    a_1 &= \frac{\ang{|x|, \cos x}}{\ang{\cos x, \cos x} }= \frac{2\int_0^\pi x \cos x dx}{\int_{-\pi}^\pi\cos^2 x dx} = -\frac{4}{\pi} \\
    a_2 &= \frac{\ang{|x|, \cos 2x}}{\ang{\cos 2x, \cos 2x}} = \frac{2\int_0^\pi x \cos 2x dx}{\int_{-\pi}^\pi\cos^2 2x dx} = 0\\
    b_1 &= \frac{\ang{|x|, \sin x}}{\ang{\sin x, \sin x}} = 0 \\
    b_2&= \frac{\ang{|x|, \sin 2x}}{\ang{\sin 2x, \sin 2x}} = 0.
\end{align*}
Note that the last two are 0 because $|x|$ is an even function and $sin(kx)$ is odd, so they are orthogonal. Thus, we have found the coefficients. \qed


\newpage
\problem{4} TODO
 \tri
\hop 
\solution
\begin{enumerate}[(a)]
    \item Let $v \in \cal{D}$ where $v \ne 0$ be an eigenvector with an eigenvalue $\lambda$. So, $Av = \lambda v$. Then, we see 
    \[\lambda \ang{v,v} = \ang{\lambda v, v} = \ang{Av, v} \ge 0.\]
    Since $\ang{v,v} >0$ we can divide both sides by it to see that $\lambda \ge 0$. Thus, all eigenvalues are non-negative. \qed
    \item We see that the boundary conditions give us a vanishing boundary term as 
    \begin{align*}
        \ang{Af, f} &= \int_0^\ell -f''\bar{f} \\
        &=  \int_0^\ell f'\bar{f}'\\
        &\ge 0.
    \end{align*}
    So, $A$ is positive. 
    \hop
    Now, we will find its eigenvalues. We know by part (a) that eigenvalues are non-negative. Constant functions give the 0 eigenvalues, so we can consider eigenvalues $\lambda^2 >0$. Then, we have $-f'' = \lambda^2 f$. The general solution to this is 
    \[f(x) = a\sin(\lambda x) + b \cos(\lambda x).\]
    As we've seen in class, the boundary conditions give us that $f$ is of the form
    \[f(x) = a\sin\del{\frac{2\pi}{\ell}nx}+b\cos\del{\frac{2\pi}{\ell}nx}.\]
    Thus the eigenvalues are 
    \[\cbr{\frac{4\pi^2}{\ell^2}n^2: n \in \ZZ}.\]
    \qed
\end{enumerate}


\newpage
\problem{5} TODO
 \tri
\hop 
\solution
\begin{enumerate}[(a)]
    \item We see that the boundary conditions give us 
    \begin{align*}
        \ang{Af, g} &= \int_0^\ell f^{(4)}\bar{g}\\
        &= \sbr{f^{(3)}\bar{g}}_0^\ell - \int_0^\ell f^{(3)}\bar{g}' \\
        &=  - \int_0^\ell f^{(3)}\bar{g}' \\
        &= \sbr{f''\bar{g}'}_0^\ell + \int_0^\ell f''\bar{g}'' \\
        &= \int_0^\ell f''\bar{g}'' \\
        &= \sbr{f'\bar{g}''}_0^\ell - \int_0^\ell f'\bar{g}^{(3)} \\
        &= - \int_0^\ell f'\bar{g}^{(3)} \\
        &=  \sbr{f\bar{g}^{(3)}}_0^\ell + \int_0^\ell f\bar{g}^{(4)} \\
        &= \int_0^\ell f\bar{g}^{(4)} \\
        &= \ang{f, Ag}
    \end{align*}
    So, $A$ is symmetric. Now, we also see by the above computation that 
    \begin{align*}
        \ang{Af,f} = \int_0^\ell f''\bar{f}'' \ge 0 
    \end{align*}
    So $A$ is positive. \qed
    \item By part (a) and by problem 4, we know all eigenvalues are nonnegative. We see that any constant function gives the 0 eigenvalue, so we can search for positive eigenvalues from now on. Let $\gamma^4$ be an eigenvalue (note any positive number can be written like this). %TODO: can these be complex?
     So, $f^{(4)} = \gamma f$. As the hint says, the general solution to this has form 
     \[f(x) = a \cosh(\gamma x) + b\sinh(\gamma x) + c \cos(\gamma x) +d \sin(\gamma x).\]
    This means 
    \[f'(x) = \gamma(a\sinh(\gamma x)+b\cosh(\gamma x) - c \sin(\gamma x) + d \cos(\gamma x)).\]
    Now, employing the boundary conditions, and assuming $\gamma > 0$,
    \begin{align*}
        f(0) = 0 &\implies &c+a &= 0\\
        f'(0) = 0 &\implies &b+d &= 0\\
        f(\ell) = 0 &\implies &a(\cosh(\gamma\ell)-\cos(\gamma\ell)) + b(\sinh(\gamma\ell)-\sin(\gamma\ell)) &= 0\\
        f'(\ell) = 0 &\implies &a(\sinh(\gamma\ell)+\sin(\gamma\ell)) + b(\cosh(\gamma\ell)-\cos(\gamma\ell)) &= 0
    \end{align*}
    So, 
    \begin{align*}
        &\frac{\sinh(\gamma\ell)-\sin(\gamma\ell)}{\cosh(\gamma\ell)-\cos(\gamma\ell)} = -\frac{a}{b} = \frac{\cosh(\gamma\ell)-\cos(\gamma\ell)}{\sinh(\gamma\ell)+\sin(\gamma\ell)} \\
        \implies& \sinh(\gamma\ell)^2 -\sin(\gamma\ell)^2 = \cosh(\gamma\ell)^2 - 2 \cos(\gamma\ell)\cosh(\gamma\ell) + \cos(\gamma\ell)^2 \\
        \implies& 0 = 2 - 2\cos(\gamma\ell)\cosh(\gamma\ell) \\
        \implies& \cos(\gamma\ell) \cosh(\gamma\ell) - 1 = 0.
    \end{align*}
    Note that if $b=0$, we can express $-b/a$ in two ways instead. We can't have both $a$ and $b$ zero since $f$ is nonzero. 
    \hop
    We see that $\gamma = 0$ also fits this equation, so we can say eigenvalues are \[\{\gamma^4:  \cos(\gamma\ell) \cosh(\gamma\ell) - 1 = 0\}.\]
    Note that $\cosh$ is non-negative and grows in both direction and $\cos$ oscillates, so $\cos(\gamma\ell) \cosh(\gamma\ell) - 1$ must have infinite roots by the intermediate value theorem. Thus, there are indeed infinite eigenvalues. \qed
    TODO: img
\end{enumerate}


\newpage
\problem{6} TODO
 \tri
\hop 
\solution
First, we show that the sequence is Cauchy. For any $\eps$, let $N >1/\eps$. We see that for any $n, m > N$, we have
\begin{align*}
    ||f_n - f_m|| &= \int_0^1|f_n-f_m|^2 \\
    &= \int_{1/2}^{1/2+1/N}|f_n-f_m|^2 \\
    &\le \int_{1/2}^{1/2+1/N}1 \\
    &= \frac{1}{N} \\
    &\le \eps.
\end{align*}
So, the sequence is Cauchy.
\hop 
Now let's show that there is no limit. Suppose for contradiction that there is some $f$ such that $f_n \to f$. Let $x > 1/2$. Now, let $\eps = x - 1/2 > 0$. Suppose $f(x) \ne 1$. Then, define 
\[c = \int_x^1 |f-1|^2.\]
Note that $c >0$. We see that for $n>1/\eps$, we have 
\[||f_n - f|| = \int_0^1|f-f_n|^2 \ge \int_x^1|f-f_n|^2 =\int_x^1|f-1|^2 = c.\]
Since $||f_n -f|| \to 0$, this is not possible. So, it must be that $f(x) = 1$. 
\hop 
By a similar argument, let $x \le 1/2$. Suppose $f(x) \ne 0$. Define 
\[c = \int_0^x |f|^2.\]
We see that for any $n$,
\[||f_n - f|| = \int_0^1|f-f_n|^2 \ge \int_0^x|f-f_n|^2  = \int_0^x|f|^2 = c. \]
This also can't be. Thus, $f(x) = 0$. So, it must be that 
\[f(x) = \begin{cases}
    1 &\text{ if } x > 1/2\\
    0 &\text{ if } x \le 1/2
\end{cases}.\]
However, this isn't continuous, so we have a contradiction. So, this sequence is Cauchy but not convergent and our space is not complete. \qed

\newpage
\problem{7} TODO
 \tri
\hop 
\solution
\begin{enumerate}[(a)]
    \item We can directly compute the geometric series. We see that 
    \[P_r(\theta) = \sum_{n=0}^\infty r^n e^{in\theta} + \sum_{n=1}^\infty r^n e^{-in\theta}.\]
    Now,
    \[\sum_{n=0}^\infty r^n e^{in\theta} = \sum_{n=0}^\infty (re^{i\theta})^n = \frac{1}{1-re^{i\theta}}\]
    and
    \[\sum_{n=1}^\infty r^n e^{-in\theta} = \sum_{n=1}^\infty (re^{-i\theta})^n = \frac{re^{-i\theta}}{1-re^{-i\theta}}.\]
    So,
    \begin{align*}
        P_r(\theta) &= \frac{1}{1-re^{i\theta}}+\frac{re^{-i\theta}}{1-re^{-i\theta}} \\
        &= \frac{1-re^{-i\theta}+re^{-i\theta} - r^2}{1-r(e^{i\theta} + e^{-i\theta}) + r^2} \\
        &= \frac{1-r^2}{1-2r\cos(\theta) + r^2},
    \end{align*}
    as we wanted. \qed
    \item For this problem, I will add a factor of $1/(2\pi)$ to the expression because otherwise I couldn't get it to work. First, we know $P_r$ is a $C^2$ function in the disk. We know convolution preserves this property, so $u$ is $C^2$ on the disk. Thus, (formally, by the inverse function theorem), $v$ is also $C^2$ on the disk. 
    \hop Now, let's check the Laplacian. Let $f(r \theta) = r^{|n|}e^{in\theta}$. For $|n| \ge 2$, we have $f_{rr} = r^{|n| - 2}(|n|-1)|n|e^{in\theta}$ and $f_{\theta\theta} = -r^{|n|}|n|^2e^{in\theta}$ and $f_r = |n|r^{|n|-1}e^{in\theta}$. We can check that 
    \[f_{rr}+\frac{1}{r^2}f_{\theta\theta} + \frac{1}{r}f_r = 0.\]
    Thus, $f$ is harmonic. So, $P_r$ is the sum of harmonic functions, so it is harmonic. Thus,
    \[\mathring{\Delta}u = \mathring{\Delta}\frac{1}{2\pi}\int_0^{2\pi} P_r(\theta - \phi)h(\phi)d\phi =\frac{1}{2\pi}\int_0^{2\pi}  \mathring{\Delta} P_r(\theta - \phi)h(\phi)d\phi = 0,\]
    where $\mathring{\Delta}$ denotes the polar Laplacian. So, we can conclude that $\Delta v = 0$. The origin does not give us problems, as said in chapter 14 of the notes.
    \hop 
    A nicer way to show the above is to notice that setting $z = re^{i\theta}$, we have $P_r(\theta) = \text{Re}\del{\frac{1+z}{1-z}}$ and then use the fact that the real part of a holomorphic function is harmonic. With this method, we do not need to worry about the origin. However, we do not know this complex analysis fact.
    \hop 
    For the last part, we must show continuity on the border. Let $h_n$ be the fourier coefficients of $h$. 
    \hop
    We see 
    \begin{align*}
        u(r, \theta) &= \frac{1}{2\pi}\int_0^{2\pi} P_r(\theta-\phi) \del{\sum_{n\in\ZZ}h_ne^{in\phi}}d\phi\\
        &= \frac{1}{2\pi}\int_0^{2\pi} \del{\sum_{n\in\ZZ}r^{|n|}e^{in(\theta - \phi)}} \del{\sum_{n\in\ZZ}h_ne^{in\phi}}d\phi \\
        &= \sum_{n,m\in\ZZ}\frac{1}{2\pi}\int_0^{2\pi} \del{r^{|n|}e^{in(\theta - \phi)}} \del{h_me^{im\phi}}d\phi \\
        &= \sum_{n,m\in\ZZ}r^{|n|}h_me^{in\theta}\frac{1}{2\pi}\int_0^{2\pi} e^{i\phi(m -n)}d\phi.\\
    \end{align*}
    For $n\ne m$, the term is 0, so 
    \[u(r,\theta) = \sum_{n\in\ZZ}r^{|n|}h_ne^{in\theta}\frac{1}{2\pi}\int_0^{2\pi}1d\phi = \sum_{n\in\ZZ}r^{|n|}h_ne^{in\theta}.\]
    We see that 
    \[\lim_{r \to 1}\sum_{n\in\ZZ}r^{|n|}h_ne^{in\theta} =\sum_{n\in\ZZ}h_ne^{in\theta} = h(\theta), \]
    so continuity on the border holds. \qed
\end{enumerate}


\end{document}