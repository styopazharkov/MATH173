
\documentclass{article}
\usepackage{amsmath, amsfonts, amssymb, amsthm} %packages for math-related
\usepackage[margin=1.2in]{geometry}
\usepackage[shortlabels]{enumitem}
\usepackage[utf8]{inputenc}
\usepackage{graphicx} % package for inserting images
\usepackage{commath} % package for things like \del, \cbr, and \sbr. These handle parentheses well.
\usepackage[mathscr]{euscript}%for \scr command 
\usepackage{../commands} %package with all of the commands for this class
\usepackage{url}
\setlength{\parindent}{0em} % so paragraphs aren't indented
\newcommand{\lcm}{\text{lcm}}
\newcommand{\Hom}{\text{Hom}}
\newcommand{\Ann}{\text{Ann}}
\newcommand{\Cc}{{C_c^\infty(\RR)}}
% ********************************************************** %
%               THINGS TO EDIT BELOW THIS LINE               %
% ********************************************************** %
\newcommand{\D}{\nabla}
\renewcommand{\d}{\partial}
\usepackage{wrapfig}
\title{\textsc{MATH 173 Problem Set 3}}
\author{Stepan (Styopa) Zharkov}
\date{April 20, 2022}
\begin{document}
\maketitle
\problem{1} TODO \tri
\hop
\solution
As the hint suggests, $u' = 0$ means by definition $u(\phi) = 0$ for any $\phi \in C_c^\infty(\RR)$. For any $\psi \in \Cc$, let $\phi_0\in \Cc$ be a bump function such that $\int_\RR \phi_0(x)dx = 1$. Let $\hat{\psi} = \psi - \phi_0\int_\RR\psi(x)dx$. We see that 
\[\int_\RR \hat{\psi}(x)dx = \int_\RR \psi(x)dx -\int_R\psi(x)dx\cdot \int_\RR\phi_0(x)dx =\int_\RR \psi(x)dx-\int_\RR \psi(x)dx = 0.\]
So, we can let 
\[\phi(x) = \int_0^x \hat{\psi}(x) dx.\]
We see $\hat{\psi}$ has compact support and is in $\Cc$ (since it is the sum of two compact support functions). Since $\int_\RR \hat{\psi}(x)dx = 0$, we know $\phi$ must have compact support as well and be in $\Cc$. Now, let $c = u(\phi_0)$. We see that by linearity of $u$,
\[u(\psi) = u(\hat{\psi}) + u(\phi_0)\cdot \int_\RR\psi(x)dx = u(\phi') + c\int_\RR\psi(x)dx =0+  c\int_\RR\psi(x)dx=c\int_\RR\psi(x)dx, \]
which is exactly what we wanted to prove. 
\qed


\newpage
\problem{2} TODO \tri
\hop
\solution
TODO
\qed


\newpage
\problem{3} TODO \tri
\hop
\solution
TODO
\qed


\newpage
\problem{4} TODO \tri
\hop
\solution
TODO
\qed


\newpage
\problem{5} TODO \tri
\hop
\solution
TODO
\qed


\newpage
\problem{6} TODO \tri
\hop
\solution
As the hint suggests, consider $v(x_0, x_n, t) = u(x_0, x_n, t) - u(x_0, -x_n, t)$. Notice that 
\[v_{tt} - c^2\Delta_xv = u_{tt}(x_0, x_n, t) - u_{tt}(x_0, -x_n, t) - c^2\Delta_{x}u(x_0, x_n, t) + c^2\Delta_{x}u(x_0, -x_n, t)  = f(x_0, x_n, t) - f(x_0, -x_n, t) = 0.\]
We also see that 
\[v(x, x_n, 0) = u(x_0, x_n,0) - u(x_0, -x_n, 0) = \phi(x_0, x_n) - \phi(x_0, -x_n)\]
and 
\[v_t(x, x_n, 0) =u_t(x_0, x_n,0) - u_t(x_0, -x_n, 0) = \psi(x_0, x_n) - \psi(x_0, -x_n). \]
So, $v$ solves the equation $v_{tt} - c^2\Delta_xv = 0$ with 0 initial conditions. We know that the only solution to this is $v=0$. Thus, we see that $u(x_0, x_n, t) - u(x_0, -x_n, t) = v(x_0, x_n, t) = 0$, so 
\[u(x_0, x_n, t) = u(x_0, -x_n, t),\]
and $u$ is even with respect to $x_n$, exactly as we wanted. \qed
\qed


\newpage
\problem{7} TODO \tri
\hop
\solution
TODO
\qed


\newpage
\end{document}